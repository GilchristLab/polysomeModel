\documentclass[a4,center,fleqn]{NAR}

%% Try and use Arial font.
%% Following instructions at (https://tex.stackexchange.com/a/23960/110759)
%% indicates one should use xelatex, not pdflatex
%% \usepackage{fontspec}

% Enter dates of publication
\copyrightyear{YYYY}
\pubdate{DD MM YYYY}
\pubyear{YYYY}
\jvolume{xx}
\jissue{xx}


%\documentclass[1p,review]{elsarticle}
%\usepackage{pdflscape}
%\usepackage{subcaption}
\usepackage[margin=3cm]{geometry} %changed from 2.5cm to 3cm so that margin notes don't get cut off.
%%\usepackage[usenames,dvipsnames]{color}
\usepackage{amsfonts}%
\usepackage{amsmath}
\usepackage{amssymb}%
\usepackage{amsthm}
\usepackage[mathlines]{lineno}
\usepackage{array}
%\usepackage{bibentry}
\usepackage{blkarray}
\usepackage{booktabs}
\usepackage{caption}
\usepackage{cuted}
\usepackage{dsfont}
\usepackage{enumerate}
\usepackage{float}
\usepackage{framed}
\usepackage{geometry}
\usepackage{graphics}
\usepackage{graphicx}
\usepackage{ifthen}
\usepackage{lineno}
\usepackage{multirow}
%\usepackage{natbib}
\usepackage{pdflscape}
\usepackage{pifont}
\usepackage{setspace}
\usepackage{subcaption}
\usepackage{tablefootnote}
\usepackage{tabularx}
\usepackage{tikz}
\usetikzlibrary{shapes,arrows,positioning,calc,fit}
\usepackage{url}
\usepackage{xargs}



%%%%% BEGIN MIKE'S COMMANDS %%%%%
\usepackage[normalem]{ulem}  %provides strikeout \sout{}
\usepackage{xspace} %%needed for mike's commands

%% Define mike's margin par
\newcommand\mmpar[1]{\marginpar{\begin{spacing}{0.7}\raggedright \singlespacing \tiny \textbf{M:} #1 \end{spacing}}}  %for notes in margin
\newcommand\rmpar[1]{\marginpar{\begin{spacing}{0.7}\raggedright \singlespacing \tiny \textbf{R:} #1 \end{spacing}}}  %for notes in margin

%%custom commands for sanity
\newcommand{\imax}{\ensuremath{{i_{\max}}}\xspace}
\newcommand{\kappaprime}{\ensuremath{\kappa^{\prime}}\xspace}
\newcommand{\tauprime}{\ensuremath{\tau^{\prime}}\xspace}
\newcommand{\mhat}{\ensuremath{\hat{m}}\xspace}
\newcommand{\mhati}{\ensuremath{\hat{m}_i}\xspace}
\newcommand{\mhatstar}{\ensuremath{\mhat^{*}}\xspace}
\newcommand{\mhatstari}{\ensuremath{\mhat^{*}_i}\xspace}
\newcommand{\mvec}{\ensuremath{\vec{m}}\xspace}
\newcommand{\mvechat}{\ensuremath{\hat{\mvec}}\xspace}
\newcommand{\mvecstar}{\ensuremath{\mvec^*}\xspace}
\newcommand{\mvechatstar}{\ensuremath{\mvechat^*}\xspace}
\newcommand{\GCD}{\ensuremath{\text{GCD}}\xspace}
\newcommand{\detA[1]}{\ensuremath{\ensuremath{\det\left[\bs{A}_{#1}\right]}\xspace}}

% Summation of vectors
\newcommand{\msum}{\ensuremath{m}\xspace}
\newcommand{\msumstar}{\ensuremath{m^*}\xspace}
\newcommand{\msumtot}{\ensuremath{M}\xspace}
\newcommand{\mfrac}{\ensuremath{f_m}\xspace}

% Expectation or MRL
\newcommand{\ribosome}{\ensuremath{\text{ribosome}}} % don't need xspace
%\newcommand{\ribosome}{\ensuremath{i}} % don't need xspace
\newcommand{\MRL}{\ensuremath{\bar{i}}\xspace}
\newcommand{\MRLs}{\ensuremath{\bar{i}\text{s}}\xspace}
%\doublespacing


%%%%% END MIKE'S COMMANDS %%%%%




%----------------------------------------------------------
%%\biboptions{round,authoryear}
\AtBeginDocument{\renewcommand{\bibname}{References}}
\renewcommand{\floatpagefraction}{0.1}
\newcolumntype{L}[1]{>{\raggedright\let\newline\\\arraybackslash\hspace{0pt}}m{#1}}
\newcolumntype{C}[1]{>{\centering\let\newline\\\arraybackslash\hspace{0pt}}m{#1}}
\newcolumntype{R}[1]{>{\raggedleft\let\newline\\\arraybackslash\hspace{0pt}}m{#1}}
\floatstyle{plain}
\restylefloat{table}
\restylefloat{figure}

%Quicklimit
\newcommand\limf[1]{\lim_{#1 \to \infty}}
%quick partials
\newcommand\p[2]{\frac{\partial #1}{\partial #2}}
\newcommand\ptwo[2]{\frac{\partial^2 #1}{\partial #2^2}}
\newcommand\mptwo[3]{\frac{\partial^2 #1}{\partial #2 \partial #3}}
%var text
\newcommand{\var}{\mathrm{var}}
%indicator
\newcommand\ind[1]{\mathds{1}_{\{#1\}}}
\newcommandx{\iton}[3][1=i,2=1,3=n]{_{#1 = #2}^{#3}}
%Norm Distrubtion
\newcommand\isnorm{\sim N(\gm,\gs^{2})}
\newcommand\issnorm{\sim N(0,1)}
\newcommand\isnormd[2]{\sim N(#1,#2)}
\newcommand\isbeta{\sim Beta(\alpha,\beta)}
\newcommand\isbetad[2]{\sim Beta(#1,#2)}
%----
\newcolumntype{A}{ >{$} r <{$} @{} >{${}} l <{$} } % A for "align"
%% (1) "r" column in math mode:          >{$} r <{$}
%% (2) no space:                         @{}
%% (3) "l" column in math mode, with 
%%     an empty subformula at the start: >{${}} l <{$}


%----


%\captionsetup[table]{skip=10pt}
%\setlength\belowcaptionskip{5pt}
\renewcommand{\arraystretch}{1.5}
\newtheorem{acknowledgement}{Acknowledgement}
\newtheorem{algorithm}{Algorithm}
\newtheorem{axiom}{Axiom}
\newtheorem{case}{Case}
\newtheorem{claim}{Claim}
\newtheorem{conclusion}{Conclusion}
\newtheorem{condition}{Condition}
\newtheorem{conjecture}{Conjecture}
\newtheorem{corollary}{Corollary}
\newtheorem{criterion}{Criterion}
\newtheorem{definition}{Definition}
\newtheorem{example}{Example}
\newtheorem{exercise}{Exercise}
\newtheorem{lemma}{Lemma}
\newtheorem{notation}{Notation}
\newtheorem{problem}{Problem}
\newtheorem{proposition}{Proposition}
\newtheorem{remark}{Remark}
\newtheorem{solution}{Solution}
\newtheorem{summary}{Summary}
\newtheorem{theorem}{Theorem}
\newtheorem{excont}{Example}
\renewcommand{\theexcont}{\theexample}
%\numberwithin{equation}{subsection}
%\numberwithin{example}{section}
\newcounter{exampleEq}
%-----------------------------------------------------------
\let\bs\boldsymbol
%----
%----
%----
\makeatletter
\@fpsep\textheight
\makeatother

%-- Flow Chart Definitions -----------------------------------------
%\tikzstyle{cloud} = [rectangle, draw, fill=red!20, inner sep=0.25em, text width=5em, text centered, rounded corners, minimum height=4em, execute at begin node=\scriptsize]
%\tikzstyle{subcloud} = [rectangle, draw, fill=green!20, inner sep=0.5em, text width=8em, text centered, rounded corners, minimum height=3em, execute at begin node=\scriptsize]
%\tikzstyle{block} = [rectangle, draw, fill=blue!20, inner sep=0.5em, text width=13em, align=center, minimum height=3em, execute at begin node=\scriptsize]
%\tikzstyle{light} = [rectangle, draw, fill=blue!10, inner sep=0.5em, text width=9em, text centered, minimum height=1em, execute at begin node=\tiny]
%\tikzstyle{io} = [trapezium, draw, fill=green!20, trapezium left angle=70, trapezium right angle=-70, inner sep=0.5em, text width=5em, text centered, minimum height=3em, execute at begin node=\scriptsize]
%\tikzstyle{decision} = [diamond, draw, fill=yellow!20, inner sep=0.05em, aspect=2, text width=4em, text badly centered, minimum height=5em, minimum width=7em, rounded corners, execute at begin node=\tiny]
%\tikzstyle{query} = [rectangle, draw, fill=red!20, inner sep=0.5em, text width=5em, text centered, minimum height=3em, execute at begin node=\scriptsize]
%\tikzstyle{prompt} = [trapezium, draw, fill=green!20, trapezium left angle=60, trapezium right angle=60, inner sep=0.5em, text width=5em, text centered, minimum height=3em, execute at begin node=\scriptsize]
%\tikzstyle{stop} = [regular polygon, regular polygon sides=8, draw, fill=red!20, inner sep=0.05em, text width=5em, text centered, execute at begin node=\scriptsize]
%\tikzstyle{go} = [circle, draw, fill=green!20, inner sep=0.5em, text width=5em, text centered, rounded corners, minimum height=3em, execute at begin node=\scriptsize]
%
%
%\tikzstyle{line} = [draw, -latex']
%\tikzstyle{noarrow} = [draw]
%\tikzstyle{container} = [rectangle, draw, inner sep=0.4em, dashed, fill=none, color=red!40]



\begin{document}
\title{Modeling mRNA Populations}
\author{%
Ricardo A. Urquidi Camacho\,$^{1}$,
Nathan Pollesch\,$^{2, 3}$
and Michael A. Gilchrist\,$^{1, 3, 4*}$%
\footnote{To whom correspondence should be addressed.
Email: mikeg@utk.edu}}

\address{%
$^{1}$Genome Science and Technology Program, University of Tennessee, Knoxville, TN 37996, USA\,
$^{2}$Department of Mathematics, University of Tennessee,  Knoxville, TN 37996, USA\,
$^{3}$Department of Ecology and Evolutionary Biology, University of Tennessee, Knoxville, TN 37996, USA\,
and
$^{4}$National Institute for Mathematical and Biological Synthesis, University of Tennessee, Knoxville, TN 37996, USA}
% Affiliation must include:
% Department name, institution name, full road and district address,
% state, Zip or postal code, country

\history{%
Received YYYY-MM-DD;
Revised YYYY-MM-DD;
Accepted YYYY-MM-DD}

%%\ead{mikeg@utk.edu}
%%\cortext[cor1]{Corresponding author}

\maketitle
\begin{abstract}
Understanding the underlying mechanisms of protein production is essential to comprehend how organisms regulated gene expression and how to co-opt these mechanisms for biotech applications. Translation and mRNA degradation are key coupled processes in the cell that regulate protein output. While both translation and mRNA degradation have been extensively studied through modeling, there is little work exploring their interaction. Here we introduce a novel coupled ODE model which integrates mRNA transcription, 5' mRNA degradation and translation. Using empirically derived parameter estimates, our model predicts the protein production from two populations of mRNA: stable and actively translating 5' capped mRNA and decapped mRNA undergoing cotranslational decay. We are able to predict gene specific capped and decapped mRNA population abundances, distribution of ribosomes loaded onto transcripts and the mean number of ribosomes associated to a transcript. Surprisingly, we find that genes with a high decapping rate (short half-life) will produce just under half their protein from decapped transcripts. Our model proves useful toward understanding a more complete view of RNA biology and protein production and acts as a steppingstone for future development in this area. 
 
\end{abstract}

% **************************************************************
% Keep this command to avoid text of first page running into the
% first page footnotes
\enlargethispage{-65.1pt}
% **************************************************************
%%\begin{keyword}
%%bioinformatics \sep mRNA population \sep protein translation \sep ribosome loading \sep ribosome count \sep polysome \sep mathematical model
%%\end{keyword}
%%\maketitle
%\newpage
%\tableofcontents


\section{Introduction}

Gene expression relies on transfer of information encoded in DNA through mRNA into a final functional form, often protein, through translation. While the central dogma represents the basic flow of genetic information, each step has multiple regulatory mechanisms adjusting gene expression. The process of translation encompasses three steps: initiation, elongation and termination and is reviewed in \citep{RN1,RN2}. In eukaryotes, translation initiation consists of a ribosome binding to the capped 5' end of an mRNA molecule. Following binding to the mRNA a ribosome proceeds to scan the mRNA until it encounters a start codon and initiates protein synthesis. Protein is synthesized through the process of elongation, as the ribosome adds one amino acid at a time to the nascent peptide chain. Finally, in the termination step, the ribosome and protein are released from the transcript. Multiple ribosomes can be on a transcript at once. Thus, polyribosomal mRNAs, or polysomes are common and often measured as a proxy for protein production by ribosome footprinting, polysome profiling and single molecule imaging of translation. 


To produce protein, translation needs a pool of capped mRNAs. The maintenance of this mRNA pool relies on a broad range of ribosome dependent and independent mechanisms \citep{RN3}. 5' decapping removes the protective 5' cap from a transcript allowing for the 5'-3' exonuclease XRN1 to degrade the transcript \citep{RN3}. If a ribosome is already present on the transcript, XRN1 trails behind performing cotranslational decay \citep{RN4}. This allows for protein to be produced from transcripts being actively degraded. Another mechanism of decay relies on 3'-5' degradation of transcripts, which does not permit for any more protein to be made \citep{RN5}. Collided ribosomes can also initiate endonucleolytic decay pathways such as nonsense mediated decay or no go decay or ribosomal quality control \citep{RN6,RN7}. The resulting 3' fragments may complete translation. The interplay between the translational machinery, mRNA degradation machinery and mRNA properties such as codon usage, secondary structure or modifications all have been reported to play a role in mRNA stability \citep{RN8, RN9, RN10}.


Both mRNA decay and translation have been explored in the literature \citep{RN11, RN12}. Some early translation modeling papers used the totally asymmetric exclusion process (TASEP) framework to study codon by codon movement of ribosomes \citep{RN13,RN14}. While powerful, TASEP is computationally expensive to run. The Ribosome Flow Model (RFM) introduced by \citep{RN15}, makes a coarse grain approximation to translation by splitting an mRNA into regions reducing computational cost at little cost to accuracy \citep{RN15}. Extensions on the RFM look at the combined translational behavior of pools of different transcript species \citep{RN16}. Other protein synthesis models take a cell wide approach to model translation \citep{RN17} providing a systems wide overview of protein production. mRNA degradation mechanisms have also been explored, either mechanistically \citep{RN18,RN19} or using a basic protein production model \citep{RN20} and are reviewed in \citep{RN21,RN22}. A combined 5' mRNA decay model has previously been explored briefly by \citep{RN22}, finding a moderate effect of decay on ribosomal load. Translation and mRNA degradation have both received ample attention in the literature, however few have explored the interaction between mRNA degradation and translation.


Here we introduce a novel coupled ODE model of mRNA polysome classes, which integrates mRNA transcription, 5' mRNA degradation and translation. Our model despite not being fit to data can demonstrate biological behavior using empirically derived parameters. The structure of the model allows for the exploration of protein production from both capped and uncapped mRNAs undergoing cotranslational decay. Moreover, we find that genes with a high decapping rate (short half-life) will produce just under half their protein from decapped transcripts. 

\section{Methods}\label{sec:description}
\subsection{Model Overview}


%A model of mRNA populations under control of translation and co-translational decay
\begin{figure*}[!ht]
\begin{center}
\includegraphics[width=120mm]{Images/Figure1_biomodel_V3.png}
\caption{Cartoon Representation of model in biological context. A) Model overview. Transcripts enter the sytem into the capped state at class 0 (no ribosomes bound). They enter the state at rate $\lambda$ through transcription. Transcripts are free to move up and down polysome classes at rates $\kappa$ for translation initiation and $\tau$ for elongation/termination. Transcripts can also be decapped and enter the decapped state at rate $\mu$. Finally, upon reaching class 0 in the decapped state transcripts are fully degraded at rate $\delta$. B) Probability of finding an mRNA in each class in the capped state. C) probability of finding an mRNA in each class in the decapped state. D) Joint probabilty of finding an mRNA in each class across each state. This reflects the total protein production potential.}
\end{center}
\end{figure*}
%\clearpage


The model captures some of the basic processes governing mRNA populations: transcript production, degradation and the process of translation (Figure 1A).
Transcripts can exist in one of two states: capped and decapped which captures the role of the 5' cap in mRNA protection and translation initiation. 
Capped transcripts are translation initiation competent, while decapped transcripts are not. 
Individual transcripts in the cell are categorized by an integer number of ribosomes (0, 1, 2, ..., \imax).
The number of ribosomes on a transcript determines that transcripts polysomal class.
The model seeks to determine how the population of transcripts of a given gene are distributed between polysome classes and capped and decapped states.

Transcripts enter into the model as defined by the transcription rate $\lambda$ into the capped state with no ribosomes,  ie. polysome class 0 ($m_0$). 
From the $m_0$ class a transcript can have two fates. 
The transcript can be decapped, thus marked for degradation at rate $\mu$ and move into the decapped class $m_0*$.
Alternatively, a ribosome can initiate translation on the mRNAs in the capped, ribosome free polysome class 0 at rate $\kappa$ and be loaded onto the transcript and move it into capped class $m_1$.
Because only one ribosomes can occupy a particular location on the mRNA at any given time and our model does not track ribosomal positions,
, we model translation intiation across polysome classes $i = 0$ to $\imax$, more generally as
\begin{equation}
  \label{eq:kappa_i}
  \kappa_i = \kappa_0 \left(1- \frac{i}{\imax}\right),
\end{equation}
where $i$ is the mRNA polysome class, \imax is the maximal ribosomal occupancy on the transcript.
We note $i/\imax$ represents, under the assumptions of a uniform distribution,  the probability a randomly chosen codon position is occupied by a ribosome.
Correspondingly, $(1 - i/\imax)$ represents probability a randomly chose codon position, such as the initiation site is unoccupied.
Because the natural units for mRNA coding sequence length in our model is the amount of space the bound ribosome takes when translating it follows that $\imax = n_{c}/9$ where $n_{c}$ is the length of the mRNA's coding sequence in codons and 9 represents the length in codons of space a single ribosome occupies.
%$\imax = n_{\text{nt}}/27$ where $n_{\text{nt}}$ is the length of the mRNA's coding sequence in nucleotides.
This attempts to account for the ribosomal density dependent effects on initiation and is called the density dependent initiation (DDI) model.


A ribosome on the transcript elongates the peptide and, in turn, terminates at a rate of $\tau$, 
Given that the length scale of our model is formulated in terms of ribosome widths but most estimates of elongation are at the scale of codons, if $\tau_c$ is the average elongation rate of an mRNA in codons, $\tau = \tau_c/9$.
As the number of ribosomes on a transcript increase, the probability of a ribosome being at the end of the transcript also increases.
Again assuming ribosomes are distributed across a transcript according to a uniform distribution, the expected ribosome termination rate on a mRNA in polysome class $i$ is simply, 
\begin{equation}
	\tau_i = \tau \frac{i}{\imax}.
\end{equation}


Capped transcripts move through rounds of translation initiation and elongation-termination and distribute along the different polysomal classes. 
From any polysome class in the capped state $m_i$ the transcript can be decapped at rate $\mu$  and move into the decapped state while maintaining the same polysomal class $m_i^*$.
Decapped transcripts can no longer initiate new rounds of translation, but allow for currently loaded ribosomes to complete translation. 
The behavior of the $m_i^*$ classes represent co-translational decay, a common method of mRNA decay in eukaryotes \citep{RN3,RN23,RN4}. 
After all ribosomes on a decapped transcript complete translation, the mRNA is in decapped class 0 and completely degraded at the mRNA clearance rate $\delta$.

The model produces two outputs. 
First, the total mRNA in either the capped or decapped state and therefore the system (Figure 1B-D). 
Second, the distribution of the mRNAs in each mRNA in each polysome class. (Figure 1 B-D).

We represent each state by a series of coupled ordinary differential equations (ODEs), one equation for the mRNA population for each polysome class. 

The functional form of the capped mRNA sub population is:
\pagebreak 
\begin{strip}
\begin{align} \label{eq:Capped_ODE}
\frac{dm_{0}}{dt} &= \lambda+ \tau \frac{1}{\imax}m_{1}-\left(\kappa_0 + \mu\right)m_{0} \\ \nonumber
\frac{dm_{1}}{dt} &= \kappa_0 m_{0}+ \tau \frac{2}{\imax}m_{2}-\left( \tau \frac{1}{\imax}+\kappa_0\left(1-\frac{1}{\imax}\right)+\mu\right) m_{1}\\ \nonumber
& \vdots & \\ \nonumber
\frac{dm_{i}}{dt} &= \kappa0 \left(\frac{i-1}{\imax}\right) m_{i-1}+ \tau \frac{i+1}{\imax}m_{i+1}-\left( \tau \frac{i}{\imax}+\kappa_0\left(1-\frac{i}{\imax}\right)+\mu\right) m_{i} \\ \nonumber
& \vdots & \\ \nonumber
\frac{dm_{\imax}}{dt} &= \kappa_0\left(1-\frac{\imax-1}{\imax}\right)m_{\imax-1}-\left( \tau +\mu\right) m_{\imax}\\ \nonumber
\end{align}
\end{strip}
Similarly, the functional form of the decapped mRNA sub population is: 
\begin{align}\label{eq:Decapped_ODE}
\frac{dm_{0}^{*}}{dt} &= \mu m_{0}+ \tau \frac{1}{\imax}m_{1}^{*}-\delta m_{0}^{*} \\ \nonumber
\frac{dm_{1}^{*}}{dt} &= \mu m_{1}+ \tau \frac{2}{\imax}m_{2}^{*}-\tau \frac{1}{\imax} m_{1}^{*} \\ \nonumber
& \vdots & \\ \nonumber
\frac{dm_{i}^{*}}{dt} &= \mu m_{i}+ \tau \frac{i+1}{\imax}m_{i+1}^{*}-\tau \frac{i}{\imax} m_{i}^{*} \\ \nonumber
& \vdots & \\ \nonumber
\frac{dm_{\imax}^{*}}{dt} &= \mu m_{\imax}^{*}- \tau m_{\imax}. \\ \nonumber
\end{align}
In closing, we note the parameters $\imax, \kappa, \mu,$ and $\tau$ likely vary between genes.


\begin{table*}[!ht]
\begin{center}
\begin{tabular}{|rp{4in}|c|c|c|}\hline
\textbf{Symbol}&\textbf{Description}&\textbf{Unit} \\\hline
State Variables & &  \\ \hline
$m_i$ & Abundance of mRNAs with a ribosome load of $i$ in capped state. & $mRNA$ \\
$m_i^*$ & Abundance of mRNAs with a ribosome load of $i$ in decapped state. & $mRNA$ \\ \hline
$p_i$ & Probability of finding an mRNA with a ribosome load of $i$ in capped state. & $mRNA$ \\
$p_i^*$ & Probability of finding an mRNA with a ribosome load of $i$ in decapped state. & $mRNA$ \\ \hline
\multicolumn{1}{l}{Model Parameters} \\ \hline
$i$ & ribosomal load index & Ribosome\\
 \imax & Maximum number of ribosomes able to bind to mRNA\\ & Ribosome \\
$\kappa_0$ & Translation initiation rate for capped mRNAs with a ribosome load of $i=0$. & $1/s$\\
$\tau$ & Translation completion rate for one ribosome & $1/s$\\
$\mu $ & Decapping rate. & $1/s$\\
$\lambda$ & Production rate of newly produced, ribosome free, and capped mRNA to the $m_0$ class. & $mRNA/s$\\
$\delta$ & Removal rate of decapped mRNA with a ribosome load of 0 from the $m_0^*$ class. & $1/s$\\ \hline 
%\multicolumn{3}{|p{450pt}|}{\footnotesize Note: $\mathbb{R}^+$ represents the non-negative real numbers and $\mathbb{Z}^{(+)}$ represents the strictly positive integers.} \\ \hline
\end{tabular}
\caption{State variables and model parameters for ODE model of mRNA populations.
Variable \imax is in the domain of non-negative integers; all other variables are non-negative real numbers.}
\label{tab:params}
\end{center}
\end{table*}


\subsubsection{Steady state solutions of the capped transcript population}
 The model solution for the capped state can be represented in the following form,
	\begin{equation} \label{eq:capped_solution}
		\mvechat=\frac{\lambda}{\mu}\vec{p}_m
	\end{equation}

Where \mvechat is a vector of the steady state mRNA abundances in each polysomal class.
\mvechat is calculated from by scaling  the vector $\vec{p}$ (where $\sum \vec(p) = 1$), which represents the steady state distribution of the mRNA across the polysomal classes, by transcript production rate $\lambda$ and the decapping rate $\mu$.
While the individual components of $\vec{p}$ are functions of $i$, \imax, the translation initiation rate $\kappa$, the elongation rate $\tau$ and $\mu$ and we could not find a closed form solution, it is worth noting that because the probability distribution of the capped population must sum to 1.
Regardless, it follows that the total abundance of the capped class \msum is,
\begin{equation}\label{eq:capped_sum}
\msum = \sum_{i = 0} ^\imax m_i = \lambda/\mu
\end{equation}

\subsubsection{Steady state solutions of the decapped transcript population}

The solution for the decapped system is dependent on the underlying distribution of the capped system and can be represented as:

\begin{align}\label{eq:decapped_abundance}
\mhatstar_0  &= \frac{\mu}{\delta}\sum_{j=0}^{\imax}m_{j} \\ \nonumber
\mhatstar_1  &= \frac{\mu}{\tau}\sum_{j=1}^{\imax}m_{j}  \\ \nonumber
& \vdots & \\ \nonumber
\mhatstar_i  &= \frac{1}{i}\frac{\mu}{\tau}\sum_{j=i}^{\imax}m_{j}  \\ \nonumber
& \vdots & \\ \nonumber
\mhatstar_{\imax}  &= \frac{1}{\imax}\frac{\mu}{\tau} \mvechat_{\imax}   \\ \nonumber
\end{align}
We can simplify the model by converting the mRNA quantity $m_{j}$ to the probability $p_{j}$ by eq. (\ref{eq:capped_solution}).
By defining $S_{j}$ as the cumulative probability from an mRNA found in class $i$ and above as,
\begin{equation}
		S_{i} = \sum_{i}^{\imax}\vec{p_{i}}
\end{equation}

Now the solution to eq. (\ref{eq:decapped_abundance}) becomes,

\begin{align} \label{eq:decapped_solution} 
\mhatstar_0  &= \frac{\lambda}{\delta}S_{0}=\frac{\lambda}{\delta} \\ \nonumber
\mhatstar_1  &= \frac{\lambda}{\tau}S_{1} \\ \nonumber
& \vdots & \\ \nonumber
\mhatstar_i  &= \frac{1}{i}\frac{\lambda}{\tau}S_{i}  \\ \nonumber
& \vdots & \\ \nonumber
\mhatstar_{\imax}  &= \frac{1}{\imax}\frac{\lambda}{\tau}S_{\imax}  \\ \nonumber
\end{align}
 Note that $S_{0}=1$ and $ S_{0} \ge ... \ge S_{i} \ge ... \ge S_{\imax}$ highlighting the fact that the steady state solution for the decapped state \mvecstar is dependant on the distribution of mRNA polysome classes of the capped state the solution vi the cumulative distribution function $S_i$. %The pattern of  the  $S$ means that $m_0 \ge m_1 \ge ... \ge ... m_i \ge ... \ge m_imax$.

The total transcript population in the decapped state $\mhatstar_{tot}$ does not have a closed form solution. However it can be summarized as follows,


\begin{align*}
	\msumstar = \sum_{i=0}^{\imax} m_{i}^{*} = \frac{\lambda}{\delta} + \frac{\lambda}{\tau}S_{1}+ \hdots \\
 + \frac{\lambda}{i \tau}S_{i} + \hdots  + \frac{\lambda}{\imax \tau}S_{\imax} 
\end{align*}

This can be further shortened to:
\begin{equation} \label{eq: marked_total_pop}
	\msumstar = \lambda(\frac{1}{\delta} + \frac{1}{\tau}\vec{S} \cdot \vec{l}	) 
\end{equation}
Where $\vec{S}$ is a vector of all the cumulative sums and $\vec{l}$ is a vector of $1,\frac{1}{2},...,\frac{1}{\imax}$ . 


To get the probability distribution of transcripts across the decapped state we can divide $\vec{m^{*}}/\msumstar$ which results in,
\begin{align}\label{eq:decapped_distribution}
	p_{0}^{*} &= \frac{1}{1 + \frac{\delta}{\tau}\vec{S} \cdot \vec{l}}	\\
  	p_{i}^{*} &= \frac{S_{i}}{i(\frac{\tau}{\delta} + \vec{S} \cdot \vec{l})}	\:\:\:\: \text{for } i=1, 2, ..., \imax
\end{align}

\subsubsection{Calculation of the total mRNA population and its distribution between capped and decapped states}
Drawing on the definitions above, the total abundance of capped and decapped mRNA polysome classes ($\msumtot$) is defined by,
\begin{equation}
	\msumtot = \msum + \msumstar = \frac{\lambda}{\mu} +  \lambda(\frac{1}{\delta} + \frac{1}{\tau}\vec{S} \cdot \vec{l})
\end{equation}

To understand how mRNA is divided  between we start with the probability of finding an mRNA in the capped state.
\begin{equation*}
	\mfrac = \frac{1}{(1  + \frac{\mu}{\delta} + \frac{\mu}{\tau}\vec{S} \cdot \vec{l})}	
\end{equation*}
Then we calculate the odds,
\begin{equation}\label{eq:odds}
	odds_{\hat{m}} = \frac{1}{\mu(\frac{1}{\delta} + \frac{1}{\tau}\vec{S} \cdot \vec{l})} \\
\end{equation}


\subsubsection{Calculating expected ribosomal load and protein production}
The expected ribosomal load for the capped or decapped state calculated, respectively, by,
\begin{align}\label{eq:Expected_ribo_load}
  \MRL =& E_{\mvec}\left(\ribosome\right) =\sum_{i=0}^{\imax}i \times p_{i} \\ \nonumber
   \text{and}&\\ \nonumber
   \MRL^* =& E_{\mvecstar}\left(\ribosome\right) =\sum_{i=0}^{\imax}i \times p^*_{i}\\ \nonumber
\end{align}
Thus, the global mean ribosomal load is,

\begin{align}\label{eq:System_ribo_load}
	\text{Total Ribosomal Load} = \mfrac\times E_{\msum} \left(\ribosome\right) +\\
          + (1-\mfrac)\times E_{\msumstar}\left(\ribosome\right)
\end{align}


%The maximum possible ribosomal load for our model is:
%\begin{equation}\label{eq:Max_output}
%	\text{Max output} = \lambda (\frac{1}{\mu}(\imax\tau) + (\frac{1}{\delta} + \frac{1}{\tau}H_{\imax}) (\frac{\imax}{(\frac{\tau}{\delta}+H_{\imax}) }\tau) )
%\end{equation}
%
%Where $H_{\imax}$ is the harmonic number for \imax.
\subsection{Numerical solution implementation in R}
Code to solve the model was written and is freely available as an R package at (\url{https://github.com/rurquidi/Ribosome}). To solve the capped subsystem of the model, the \texttt{solve.tridiag} function from \texttt{limSolve package} (v1.5.6) \citep{RN41}. The decapped solution was obtained by using the capped solutions into eq. (\ref{eq:decapped_solution}). Utility functions, plots and statistics were created using  \texttt{R} (v 3.6) \citep{RN43}, and  \texttt{data.table} (v1.14.0) \citep{RN42}. 
		
\subsection{Data Sources}

In order to biologically contextualize and illustrate our model's behavior, we use on parameter values derived from the literature.
Protein lengths were extracted from the Ensembl (version 109) and Ensembl plants (version 56) respectively \citep{RN26,RN25,RN24}.  
The range of \imax is determined from the distribution of protein lengths obtained from brewer's yeast (\textit{saccharomyces cerevisiae}) and \textit{Arabidopsis thaliana}. The range of \imax is  48 (36) (mean (SD)) for yeast and 47 (30) for Arabidopsis (Figure 2A and C).
The decapping rate between the capped and uncapped system was estimated from the protein half-lives from Presnyak (2015) for yeast (Figure 2B) and Sorenson (2018) for Arabidopsis (Figure 2D) \citep{RN27,RN28}.
We estimated gene specific $\mu$ from the half lives with the following:
	\begin{equation*}
		\mu_i = \frac{\ln(2)}{t_{1/2_i}}
	\end{equation*}

Where $t_{1/2}$ is the half-life. The resulting range of $\mu$ is from $1.3 \times 10^-3 (1.8 \times 10^-3)$ for yeast and $1.7 \times (10^-4 \pm 2 \times 10^-4)$ for Arabidopsis. 

Translation initiation and average elongation rates ($\kappa$ and $\tau_c$) were obtained for Yeast from Duc and Song 2018 \citep{RN13}. 
%In Duc and Song 2018, the authors used 850 highly translated transcripts from the ribo-seq dataset from Weinberg 2016. They employed a TASEP model to estimate the initiation rates and correct the empirical elongation rates from the footprint distributions. 
We calculated an average gene specific elongation rate from the corrected elongations rates. We scale the each gene specific initiation rate by dividing it by the gene specific elongation rate.
\begin{equation}
	\text{initiation to elongation ratio} = \kappa' = \frac{\kappa}{\tau}
\end{equation}

This simplifies the model behavior to one generalized parameter with a unique response (Figure 2E).  The initiation to elongation ratio ranges from 0.1$s^{-1}$ to 0.001$s^{-1}$.

The transcriptomic results from Weinberg 2016 are included in Figure 2F. In short, reads per kilobase million from Weinberg were further converted into a log10 fold change based on the median expression level \citep{RN29}. Figure 2F shows that the absolute range of transcriptional expression ranges just under 5 orders of magnitude. Because transcription rate $\lambda$ acts as a scaling factor throughout the model and does not affect the distribution of the ribosomes, for simplicity we set $\lambda$ = 1.

Because the mRNA clearance rate $\delta$ only determines the accumulation of transcripts in the  $m_0^*$  class, for simplicity, we set $\delta >> \tau$ and thus  $m_0^* \sim 0$.
 
The empirical mean ribosomal load (\MRL) for the 850 genes in Duc and song 2018 was calculated from the mRNA-seq read per kilbase million (mRNA RPKM) and the ribo-seq footprints (RPF RPKM) from Weinberg 2016 \citep{RN29}. The following equation was used.
\begin{equation}\label{eq:MRL}
	\MRL_i = \frac{RPF\: RPKM_i}{mRNA\: RPKM_i \times \frac{200}{length mRNA_i}}
\end{equation}

Where the gene specific scaling factor $\frac{200}{length mRNA_i}$ corrects for the bias in read counts due to longer transcripts producing more fragments. The value 200 arises from the average fragment size of a library prep and can be adjusted according the experimental method used.

\begin{figure*}[!ht]
\begin{center}
\includegraphics[width=120mm]{Images/2023-07-04_parameter_histograms.png}
\caption{Histograms of empirical values of model parameters. A) Yeast protein lengths. B) Yeast half-life C) Arabidopsis Protein Lengths. D) Arabidopsis Half-Life. E) Yeast Scaled elongation rates (Translational initiation rate/average translation elongation rate) on a per gene basis. F) Log 10 Fold Changes between all transcripts compared ot the median transcript expression in yeast. }
%\centering Red: de_s class, Green: capped class, Blue: Total= capped+decapped classes}
\end{center}
\end{figure*}


% It could be futher extended to include more degradations mechanisms see note below
% $\delta\ might be more complex with a slightly different model design. In the current model formulation it mimics the primary (but not only) degradation with the cell, the 5' decapping and subsequent VCS /XRN1 5'-3' cotranslation degradation system. This ignores 3 other systems the two 3'- 5' pathways (CCR4-NOT3) and exosome, and the final slew of endodegradation pathways (NGD, NSD, NMD, ribothrypsis, and silencing).
%Additionally, degradation of the bulk of the mRNA transcript might not be a process specific to each transcript (ie. due to coding sequence, I would have to look this up. Also might be testable with data fitting).


\section{Results}

\subsection{Model presents steady state distribution of mRNAs across polysome classes for each initiation to elongation ratio}

The model predicts the abundances of the mRNA in the capped and decapped states as well as the mRNA's distribution across polysome classes.

Steady state distribution of the capped mRNA polysome classes

The analytical steady state solution eq. (\ref{eq:capped_solution}) is composed of a vector of probabilities $\vec{p_m}$ that an mRNA is in polysome class $i$ and a scaling term (the transcription rate $\lambda$ divided by the decapping rate $\mu$). 
This solution highlights two separate roles of $\mu$. 
First, the scaling term $\lambda / \mu$ determines total mRNA abundance in the capped state. 
Second, the vector of probabilities is a function of the initiation rate $\kappa$, the elongation/termination rate $\tau$ and $\mu$ and is independent of $\lambda$.
Figure 3A shows the mRNA distribution in the capped state for four different initiation to elongation ratios $\kappa'$, for a protein of median length (\imax of 39) with a low decapping rate ($2\times10^{-4} /s$).
To summarize the model results across a range of parameters a heatmap where each row is the steady state distribution of mRNA at a particular $\kappa'$ is shown (Figure 3B).
The steady state density in the capped system is bounded at class 0 and class \imax 
When $\kappa'<<\tau$, the steady state distribution concentrates at low $i$ near the $i=0$ boundary.
As $\kappa'$ increases, the steady state distribution moves towards higher $i$ and can be roughly approximated by a truncated gaussian. 


\begin{figure}[!ht]
%\begin{center}
\includegraphics[width=75mm]{Images/2023-07-04_Unmarked_slices.png}
\caption{mRNA distribution in capped state. A) Distribution profiles for four scaled initiation values i) $2\times 10^{-1}$ ii) $1.03\times 10^{-1}$ iii) $3\times 10^{-3}$ iv) $2\times 10^{-4}$ B) Heatmap of model output across a range of scaled initiation values. Lines represent slice represented in A). Results produced with \imax of 39 and a low decapping rate of $2\times10^{-4}$  (99\textsuperscript{th} percentile). Color bar shows probability of finding mRNA in particular polysome class.}
%\centering Red: decapped class, Green: capped class, Blue: Total= capped+decapped classes}
%\end{center}
\end{figure}


Steady state distribution of the decapped mRNA polysome classes
The decapped state is again scaled by the transcription rate $\lambda$.
The analytical steady state solution eq. (\ref{eq:decapped_solution}) can be understood in two parts: $m_0^*$ and the remaining $m_i^*$ for $i>0$.
$m_0^*$ is solely determined by the ratio of the mRNA production rate to the clearance rate $\lambda / \delta$.
The remaining decapped polysome classes depend on the elongation/termination rate $\tau$ and the distribution of \mvechat.
 Figure 4A shows the mRNA distribution in the decapped state for $i>0$ under the same conditions as Figure 3, for a median length protein with a low decapping rate ($2\times10^{-4}$) and the full range of $\kappa'$ are shown in the heatmap in Figure 4B.
The mRNA distributions are greatest in the $i=1$ polysome class and are monotonically decreasing. 
 

\begin{figure}[!ht]
\begin{center}
\includegraphics[width=75mm]{Images/2023-07-04_Marked_slices.png}
\caption{mRNA distribution in decapped state. A) Distribution profiles for four scaled initiation values i) $2\times 10^{-1}$ ii) $1.03\times 10^{-1}$ iii) $3\times 10^{-3}$ iv) $2\times 10^{-4}$ B) Heatmap of model output across a range of scaled initiation values. Lines represent slice represented in A). Results produced with \imax of 39 and a low decapping rate of $2\times10^{-4}$  (99\textsuperscript{th} percentile). Color bar shows probability of finding mRNA in particular polysome class.}
%\centering Red: decapped class, Green: capped class, Blue: Total= capped+decapped classes}
\end{center}
\end{figure}


Steady state joint distribution for the combined capped and decapped mRNA classes
The full model combines the mRNA distributions from the capped and decapped states, which is what is measured by most biological assays.
Figure 5A shows the mRNA distribution in the full model for three values of $\kappa'$, for a median length protein with a low decapping rate ($2\times10^{-4}$) and the full range of $\kappa'$ are shown in the heatmap in Figure 5B.
The system is unimodal at low $\kappa'<0.01$ when the capped and decapped distributions overlap around low $i$ (Figure 5A mid and low). 
As $\kappa'>0.01$ increases, the full distribution becomes bimodal (Figure 5A high).
The peak at low $i$ representing the decapped distribution, and the higher gaussian peak representing the capped distribution. 


\begin{figure}[!ht]
\begin{center}
\includegraphics[width=75mm]{Images/2023-07-09_Figure1_DIIvsDDI_medianlength_low_marking_with_labels.png}
\caption{mRNA distribution in the full model. A) Distribution profiles for three scaled initiation values low) $1\times 10^{-3}$ mid) $2\times 10^{-2}$ and high) $1\times 10^{-1}$ B) Heatmap of model output across a range of scaled initiation values. Lines represent slice represented in A). Results produced with \imax of 39 and a low decapping rate of $2\times10^{-4}$  (99\textsuperscript{th} percentile). Color bar shows probability of finding mRNA in particular polysome class.}
%\centering Red: decapped class, Green: capped class, Blue: Total= capped+decapped classes}
\end{center}
\end{figure}


\subsection{Decapping reduces ribosomal loads and shifts mRNA from the capped polysome classes to the decapped polysome classes}

To explore the role of mRNA stability on mRNA populations we varied the decapping rate $\mu$ from the 1\textsuperscript{st}, 50\textsuperscript{th} and 99\textsuperscript{th} percentile values as determined from  \citep{RN27}.
As $\mu$ increases the distribution of mRNAs changes in two ways. 
First, there is shift to lower polysome classes in the capped state (Figure 6).
This is due to the mRNAs leaving the capped state at a higher rate and driving the equilibrium towards lower ribosomal loads. 
Secondly, as $\mu$ increases, a larger proportion of the mRNA is found in the decapped state. This is further explored later. 


\begin{figure*}[!ht]
\begin{center}
\includegraphics[width=140mm]{Images/2023-07-09_Figure2_Marking_Rate_range_medianlength_with_labels.png}
\caption{Higher decapping rates $\mu$ reduce ribosome load in the capped system in yeast.  A)  Heatmaps for the full model. Left) low $\mu$ ($2\times 10^{-4}$ /s) Center) median $\mu$ ($1.7\times 10^{-3}$ /s) Right) $\mu$ ($5.7\times 10^{-3}$ /s)  B) individual density profiles for low (0.001), mid (0.01) and high (0.1) scaled initiation values for each $\mu$ All results calculate with \imax = 39.}
%\centering Red: decapped class, Green: capped class, Blue: Total= capped+decapped classes}
\end{center}
\end{figure*}


Plants and other multicellular eukaryotes tend to have lower translation initiation $\kappa$ and elongation rates $\tau$ as well as slower cell division when compared to single celled organisms such as yeast.
This is highlighted by the current gold standard study of mRNA half-lives in the model organism \textit{Arabidopsis thaliana}, where the estimated decapping rates $\mu$ estimated are ten to one hundred times lower than those in yeast. 
To explore the effect of lower $\mu$ in Arabidopsis (range: $7.7 \times 10^{-6}$ to $1 \times 10^{-3}$ ) vs yeast (range $2 \times 10^{-4}$ to $5.7 \times 10^{-3}$ ), we ran the model using the same initiation to elongation ratios as in yeast, the median Arabidopsis \imax of 41. 
As expected, the lower$\mu$ results in an mRNA distribution at higher $i$ (Figure 7) and are mostly in the capped polysome classes (Figure 10).


\begin{figure*}[!ht]
\begin{center}
\centering
\includegraphics[width=140mm]{Images/2023-07-09_Figure2_At_Marking_Rate_range_medianlength_with_labels.png}
\caption{Low decapping rates $\mu$ in Arabidopsis result in polysome class distributions centered at higher $i$ and more mRNA abundance in the capped vs decapped polysome classes.  A)  Heatmaps for the full model. Left) low $\mu$ ($7.7\times 10^{-6}$ /s) Center) median $\mu$ (1$\times 10^{-4}$ /s) Right) $\mu$ ($1\times 10^{-3}$ /s) B) individual density profiles for low (0.001), mid (0.01) and high (0.1) scaled initiation values for each $\mu$. All results calculate with \imax = 41.}
%\centering Red: decapped class, Green: capped class, Blue: Total= capped+decapped classes}
\end{center}
\end{figure*}



\subsection{Decapping rate and ribosomal load determine ratio between capped and decapped states}
 As shown in previous results, higher decapping rates $\mu$ lead to lower \MRL in the capped state and increase mRNA abundance in the decapped state.
Using eq. (\ref{eq:odds}), we can determine how much of the mRNA population is in the capped state.
We produced output across all scaled initiation values $\kappa'$ and under the 1\textsuperscript{st}, 50\textsuperscript{th} and 99\textsuperscript{th} percentiles for decapping rates in both yeast and Arabidopsis (Figure 8). 
We note two patterns. First as the $\kappa'$ increases, so does the amount of mRNA in the decapped class  \mvechatstar increases. 
Secondly and similarly, higher $\mu$ shifts mRNA population from the capped state to the decapped state as previously seen in Figures 6 and 7.  



%Main results: 
%Link our mean output and the results of TASEP ribosome flow model.
%because short genes experience inteference at short loads becomes accentuated because turnover rate is high
%mention heterologous expression due to codon usage bias
% mean initiation rate to mean elongation rate ratio and shorten it to the Initiation elongation ratio 

%Point for discussion. By varying the granularity of the Ribo flow model it would be possible to fit the model for distinct regulatory regions. In this idea I'm trying to communicate that there might be distinct mRNA regions that behave like different transcripts. Some transcripts might be very granular like every 10 codons has a distinct behavior, But i would imagine that any one major bottleneck in a transcript would make it

\begin{figure}[!ht]
\begin{center}
\includegraphics[width=70mm]{Images/2023-07-28_logodds.png}
\caption{Percentage of mRNA in the capped state for a range of decapping rates in yeast ( low $\mu$ ($2\times 10^{-4}$ /s), median $\mu$ ($1.7\times 10^{-3}$ /s), high $\mu$ ($5.7\times 10^{-3}$ /s)) and Arabidopsis( low $\mu$ ($7.7\times 10^{-6}$ /s, median $\mu$ (1$\times 10^{-4}$ /s), high $\mu$ ($1\times 10^{-3}$ /s)). }
%\centering Red: decapped class, Green: capped class, Blue: Total= capped+decapped classes}
\end{center}
\end{figure}

\subsection{At steady state protein production is scales with coding sequence length \imax}
At steady state the \MRL increases with coding sequence length and begins to assymptote at high initiation to elongation ratio $\kappa'$ (Figure 9). %\rmpar{Mike, you were right. There is a DDI effect on mean ribosomal density. After looking into it deeper I found the old findings completely wrong. Created this new set of results to properly explain the \MRL behavior}
Yet the capped state \MRL is always greater than the decapped \MRL. The \MRL of the whole system is defined by both capped and decapped \MRLs in eq. (\ref{eq:Expected_ribo_load}) as well as the transcript abundance in each state as shown in eq. (\ref{eq:System_ribo_load}).
As protein length \imax increases, mRNAs enter the decapped state at higher polysome classes. 
Therefore ribosomes take longer  to clear the mRNA, and thus increase the contribution from the decapped state. 
For $\kappa'$= 0.1, the percentage of the mRNA in the capped class is  99\% 78\%  and 35\% for \imax of 4, 39 and 194 respectively.
The \imax dependence is captured in the 1/$\tau$ term in eq. (\ref{eq:odds}).
  
\begin{figure*}[!ht]
\begin{center}
\includegraphics[width=120mm]{Images/MRL.png}
\caption{The mean ribosomal density on a transcript is dependent on coding sequence length. \MRL per transcript is higher for longer transcripts. A) Capped state B) Decapped state C) full model. Yeast parameters were used \imax =  4 (1\textsuperscript{st} percentile),  39 (50\textsuperscript{th} percentile), 194 (99\textsuperscript{th} percentile), low decapping rate ($2.2\times 10^{-4}$ /s), over the full scaled initiation range 0.0001- 0.5.}
%\centering Red: decapped class, Green: capped class, Blue: Total= capped+decapped classes}
\end{center}
\end{figure*}

\subsection{Decapped state can be a significant source of protein production} \rmpar{may now be superfluous}
%For the capped state rate on is rate off at equilibrium. And by using the initiation rate, it is the same as the protein production rate per transcript. Same applies for %\tau_0$/9*\imax as this is equal to the initiation rate.
%For the decapped state, the equilibrium exists between the inflow of marked transcripts at in a particular distribution and the same %\tau_0$/9*\imax parameter. Or not really. There is really no "on" rate, just an off rate.
%in the capped sytem the on off rate directly relates to the ribosomal density/load on a transcript and it is a decent ad hoc measure of relative expression. 
%Decapped is more complex. The mRNAs enter at rate $\mu$, but these don't carry the same number of ribosomes with them. Unlike $\kappa$ they are not the loading rate of one ribomsome. Is it the mean riboload of the capped system times $\mu$? No, the mean underestimate the contribution of the higher classes needing to percolate down into the lower classes. The decapped class steady state has no simple way of calculating it's steady state load from the parameters.
Protein production rate (PPR) is a function of full model \MRL $\times \tau$ and plots normalized to highest possible protein output are shown in Figure 10.
As the decapping rate $\mu$ increases it reduces the capped and uncapped \MRL as well as shifting transcript abundance to the decapped state (Figure 10A).
Each of the three cases in Figure 10 A, has been broken down into the PPR contributions from the capped and decapped states (Figure 10 B-D). 
A surprising finding from out model is that when $\mu$ is high ($5.7\times 10^{-3}$ /s), 41\% of all protein production can arise from the decapped state (Figure 10 D).
The reason behind this is despite the the \MRL of the decapped state being lower than the capped state as scaled initiation rises, the shift of mRNA from the capped polysome classes to the decapped polysome is enough to offset the lower $\MRL^*$
%\ref{eq:Max_output}


\begin{figure*}[!ht]
\begin{center}
\includegraphics[width = 120mm]{Images/2023-07-17_Protein_Production_v2.png}
\caption{Estimated average protein production in yeast.  A) Protein production across different decapping rates  low $\mu$ ($2.2\times 10^{-4}$ /s), median $\mu$ ($1.7\times 10^{-3}$ /s), high $\mu$ ($5.7\times 10^{-3}$ /s). Total protein production is normalized to the maximal protein production across all parameters. B-D) Contribution of capped and decapped states to total protein production. B) Low decapping C) Medium decapping D) High decapping. }
%\centering Red: decapped class, Green: capped class, Blue: Total= capped+decapped classes}
\end{center}
\end{figure*}


\subsection{Model validation}

Gene specific \MRL eq. (\ref{eq:MRL}) were calculated for the genes analyzed in Duc and Song 2018 and compared to the empirical \MRL calculated from raw data from Weinberg 2016 (Figure 11).
Model predictions of \MRL showed a significant positive correlation to empirical \MRLs.
This result is impressive as the model performs well despite no model fitting being performed.
Model performance is further corroborated with single molecule imaging analyses.
Rescaling ribosome abundances from each single molecule study to an \imax of 39 results in loads of 1, 2.4 and 4 ribosomes from \citep{RN30}, 3 ribosomes \citep{RN31}, 4 ribosomes \citep{RN32} and 1.6 \citep{RN33}.
The single molecule measurements agree with model predictions, as most transcripts have a MRL of 0-6 for median decapping rate ($1.7\times 10^{-3}$ /s) and  $\kappa' < 0.01$ and \imax of 39.
%  \item All of which align with low to mid initiation to elongation ratio \MRL predictions. 
  %\item Finally, mRNA distributions for the full model agree with signal from polysome traces (Lokdarshi 2020, Dasgupta 2023).



\begin{figure}[!ht]
\begin{center}
\centering
\includegraphics[width=75mm]{Images/Duc_Song_vs_model_log.png}
\caption{Predicted mean ribosomal loads coincide with observed mean ribosomal loads from Weinberg 2016. Using the 850 genes from Duc and song 2018, decapping rates from Presnyak 2015 were mined. Gene specific \MRL were calculated and compared to the empircal \MRL. Spearman's $\rho$ was calculated and found to be significant, pvalue $<10^{-16}$}
%\centering Red: decapped class, Green: capped class, Blue: Total= capped+decapped classes}
\end{center}
\end{figure}



\section{Discussion}

In this study we develop, analyze, and validate a novel coupled ODE model of mRNA polysome classes %\mmpar{Terminology: do we like `mRNA polysome populations'?}\rmpar{Yes}
which includes the contributions of mRNA transcription, the initiation, elongation (implicitly), and termination of translation as well as mRNA degradation through 5' decapping and cotranslational decay.
%\rmpar{Need to be specific. \textbf{M:} So be specific.}
%\rmpar{Yes our model is a simplified description of protein synthesis. It contains no new aspects of the process of protein synthesis itself. It does include the interaction between two commonly studied processes. Translation and mRNA degradation. \textbf{M:} So incorporate this fact into the text here or below. Also, isn't the way the model treats ribosomes on the transcripts (uniform as opposed to flux or explicit) novel? \textbf{R:} Done. I don't recall seeing anymodel that treats the ribosomal distribution as uniform. Nor one that looks at it through the length of polysomal classes.}
%\mmpar{(Add relevant equation and figure references to points below)}
\subsection{Model Formulation \& Structure}

Although our model is only a very simplified description of the mRNA polysome population and, in turn, protein translation, it studies the interaction of protein translation with the process of mRNA degradation, a topic which has been underexplored \citep{RN11}. The process of translation is dependent on the underlying population of capped, translationally competent mRNAs. However, empirical measurements suggest that $\sim 12\%$ of transcripts are undergoing co-translational decay \citep{RN4}. To undergo co-translational decay, the 5' cap has to removed and exonucleases trail behind the last loaded ribosome on a transcript, processing codons as they exit behind the ribosome. 5' decapping is a common pathway in many organisms and accounts for decay; for example $\sim 68\%$ of Arabadopsis genes are \citep{RN28}.
%  \mmpar{I uncommented this point, can you reconcile it with the previous sentence? \textbf{R:} Done. The first is a statement of the proportion of the population undergoing decay. The second the proportion of the genome that preferentially uses this pathway for decay.} 
Our model includes 5' mRNA decapping followed by cotranslational decay, permiting the analysis of the decapped mRNA state (called degratome in \citep{RN34}), changes in mean number of ribosomes per transcript \MRL for capped and decapped states and the contribution of cotranslational decay to protein production.

In addition to being more biologically realistic, structuring the mRNA population by its polysome classes (ribosome load) and the status of its 5' cap allows us to understand how the the rates mRNA production $\lambda$, decapping $\mu$, protein elongation $\tau$, and the clearance rate $\delta$ of decapped and ribosome free mRNAs $\mhatstar_0$ shape the steady state distribution of a gene's mRNA population across polysome classes and capping state (Figures 3-5).
Overall, we find that 


Analytical and numerical solutions show transcription rate $\lambda$ acts as a scaling factor such that the abundances
  % \mmpar{abundance or density? \textbf{R:} Abundance. $\lambda$ scales the density to a number of transcripts. }
  of all of the mRNA polysome classes are proportional to $\lambda$.
  In other words, the total abundance of the capped and decapped mRNA polysome classes $\mvechat$ and $\mvechatstar$ are simply proportional to $\lambda$ (see (\ref{eq:capped_sum}) where $\sum_{i = 0} ^\imax m_i = \lambda/\mu$ and (\ref{eq: marked_total_pop}), respectively).
  The fact that the abundance of the entire capped and decapped mRNA polysome classes are proportional to the transcrption rate $\lambda$ is consistent with intuition, as $\lambda$ increases, so does the abundance of both the capped and decapped populations.
  Similarly, the fact that the abundance of the capped mRNA polysome classes declines as an inverse funcdtion of the decapping rate $\mu$ is also consistent with intuition.
  Because it is the ratio of $\lambda$ and $\mu$, rather than their individual values, that determine the size of the capped mRNA polysome classes, our model indicates that there will be an infinite set of transcription $\lambda$ and decapping rates $\mu$ that can result in the same population size of capped mRNA polysomes.
  All else being equal, this result suggests that these rates could vary greatly between genes with similar abundances.


The fact that changes in the mRNA transcription rate  $\lambda$ only scales, rather than shapes, the relative distribution of mRNA polysome classes allows us to turn our focus to how the remaining model parameters, $\kappaprime = \kappa/\tau$, $\mu$, and $\delta$ alter the \emph{relative} distribution of the capped and decapped mRNA polysome classes \mvechat and \mvechatstar, respectively.

For example, focusing on the relative distribution of the capped mRNA polysome classes $\mvechat$, our model indicates that it is the ratio of scaled translation initiation $\kappaprime = \kappa/\tau$ to decapping $\mu$ which determines the distribution \mvechat, and thus its mean ribosome load  (Figure 3).
   % \mmpar{Starting in the results, let's use \kappaprime to represent $\kappa/\tau$}
    For example, when the initiation rate is much less than the decapping rate $\kappaprime/\mu \ll 1$, the distribution of capped mRNA polysome classes $\mvechat$ is greatest in the ribosome free polysome class $i=0$ and declines rapidly with with ribosome load $i$.
    As $\kappaprime/\mu$ increases, the distribution of capped mRNA polysome classes shifts away from the lower bound of $i = 0$ appears to follow a truncated gaussian distribution.
    In contrast, it is only at very high and generally unrealistic values of $\kappaprime/\mu$ (i.e.  when $\kappa>>\mu or \tau$, so that $\kappaprime/\mu > 10$) do we see the peak of the distribution of capped mRNA polysome classes approach $\imax$.
 %   \mmpar{Is my revision correct? \textbf{R:} yes}



Shifting our focus to the relative distribution of the decapped mRNA polysome classes \mvechatstar, our model provides a number of important insights .
Surprisingly, in the special case of the decapped, ribosome free mRNA class $\mhat_0^*$, we find its abundance is decoupled from the dynamics of the rest of the population.
  This decoupling has a number of important implications.
  For example, the steady state abundance of $\mhat_0^* = \lambda/\delta$ and, thus, depends only on the ratio of the mRNA transcription rate $\lambda$ to the mRNA clearance rate $\delta$ (equation \ref{eq:decapped_solution}).
  If the transcription rate $\lambda$ of new, capped, but ribosome free mRNAs $\mhat_0$ is substantially lower than the per capita mRNA clearance rate of decapped, ribosome free mRNAs $\delta$, such that  $\lambda \ll \delta$, then our model predicts that there will be few mRNAs in the $\mhat_0^*$ class ($\mhat_0^* \ll 1$).
  Because $\mhat^*_0$ has no impact on the rest of the mRNA population, this result allows us to greatly simplify our analysis further since we need not consider $\mhat_0^*$ nor the parameter $\delta$.

Focusing now on the steady state abundance of the ribosome occupied decapped mRNA polysome classes, i.e.  $\mhat_i^*$ where $i > 0$, we find that the distribution of $\mhat^*_i$  depends on the gene specific ribosome elongation rate $\tau$ (where `elongation`  includes the ribosome's reading of the mRNA's stop codon) and the distribution of capped mRNA $\mhat_i$ (Figure 4).
\label{item:protein_production} This finding implies that because the density of $\mhat^*_i$ monotonically decreases with $i$, the distribution of decapped mRNA polysome classes is skewed and dominated by lower polysomal classes.
  This is monotonic decline coupled with the fact that the decapped ribosome free polysome class $\mhat_0$ does not contribute to protein production, implies that \MRL of the decapped mRNA polysomes is must be less than \MRL the of the capped mRNA polysomes.
  Thus, while the decapped class does contribute to protein production, substantially under low decapping ($\mu> 5.7 \times 10^{-3}$) or large \imax, its contributution to the mRNA population's protein production will always be less than  $50\%$.



The full model combines the distributions of the capped and decapped states and is equivalent to mRNA population that is often measured in translational assays.
The full model distribution is strictly unimodal when initiation occurs and $\kappaprime/\mu \ll 1$ (Figure 6 and 7) and the majority of transcripts are in the capped state ($\sum \mvechat >> \sum \mvechatstar$), effectively .
The distribution is also unimodal when the $\kappaprime << 0.01$, meaning that the \MRL is low and near the $i=0$ bound. 
In all other cases the distribution of polysome classes is bimodal. The high peak arises from the capped state, while the small novel  peak comes from the decapped states. 
To the best of our knowledge, the only prior model that combines cotranslational decay and translation, agrees with our findings of decreased \MRL when decapping occurs \citep{RN22}. However, they didn't explore a range of biologically relevant parameters nor did they analyze the contributions from the capped and decapped polysome classes separately.

The model formalizes the interplay between mRNA decapping $\mu$ and initiation elongation ratio $\kappa/\tau$ and its effect on protein production.
By using eq.  [\ref{eq:System_ribo_load}] we can estimate protein production rate.
Shifts in mRNA between capped and decapped states as well as changes in \MRL control protein production.
As expected, increasing $\mu$ raises the proportion of decapped transcripts \mvechatstar compared to capped transcripts \mvechat.
Increasing ratio of initiation to elongation rates $\kappa/\tau$ also results in an increase of decapped mRNAs \mvechatstar.
As $\kappa/\tau$ increases the \MRL of the capped population \mvechat increases, transcripts enter the decapped state at higher polysomal classes and thus take longer to reach $m_0^*$.
At high $\mu$ this shift in transcripts is enough to shift more protein production to the decapped polysome classes, but not enough to overtake the capped protein production.
One final consideration is the assumption that the mRNA clearance rate $\delta >> \lambda$, and therefore $\mhatstar_0$ will be negligible and won't affect protein production at the population scale. If $\mhatstar_0$ is small our current results act as an upper bound of the protein production contribution from the decapped class. 


A unique property of our model is that it can differentiate the capped and decapped states individual contributions to protein production. 
A surprising prediction from our model is that genes with  high decapping rates (e.g. $\mu  \sim 5 \times 10^{-3}$ or a half-life of $\sim 120 \sec$) has almost (but never more than) half of its protein production coming from the decapped mRNA polysome classes (Figure 10).
The high protein production from the decapped polysome classes suggests that mRNAs with high decapping rate can produce more protein than would be expected based on this $M_{tot} = \frac{\lambda}{\mu}$ alone. 


\subsection{Model Validation}

In addition to studying the general behavior of our model, we validate this behavior using empirically based parameter values.
In general, we find that our model's predictions of mRNA distributions, when parametrized with biologically relevant values, are highly consistent with a wide range of empirical data.
For example, we predicted \MRL using empirical values for initiation to elongation rates ratio $\kappa'$ \citep{RN13} and decapping rate $\mu$ \citep{RN27} and compared them to the empirical \MRL from \citep{RN29} and found a strong correlation despite having performed no additional model fitting. This supports the idea that the model is a useful representation of the complex processes underlying protein production.

The $\kappa'$ estimates utilized are only for highly translated genes (16\% of alll detected genes), and most others would fall in a range of $\kappa'<0.01$ \citep{RN13}.
Taking the overall low $\kappa'$ values, our model predicts that a median length protein of \imax =39 (351aa) would have 10 or fewer ribosomes loaded, which agree with the predominance of low polysomes (<10) seen polysome gradient traces \citep{RN35, RN36}. 
By the same logic, we find that single molecule measurements of translation \citep{RN30,RN32,RN33,RN31} (Morisaki 2016, Yan 2016, Wang 2016, Wu 2016, Section 3.6) all fall in the same low polysome range.
Finally, the fraction of mRNA predicted in the capped and decapped class  are consistent with population wide estimates (\citep{RN4}, Figure 10). 


\subsection{Model limitations, extensions and future work}

Our model's assumptions about the process of mRNA decapping, the continued translational competence of transcript ribosomes bound prior to decapping, and degradation of mRNA solely from the decapped and ribosome free class $m_0^*$  closely resembles the biological process of co-translational mRNA decay. While the existence of co-translational mRNA decay is well established \citep{RN4,RN28}, other mechanisms exist with different outcomes for translation. 3' decay prevents bound ribosomes from completing translation and would send all transcripts into the $m_0^*$ class . Mechanisms utilizing endonucleolytic decay due to no go decay or nonsense mediated decay would potentially allow ribosomes downstream of the cleavage site to terminate but not those upstream \citep{RN38,RN2}. Thus,  the site of the endonucleolytic decay a transcript in $m_i$ would end up in $m_{j^*}$, where $j < i$.

Currently there is debate about the contributions of the protective effects of ribosome association vs. ribosome stalling to mRNA transcript stability.
While our model currently does not include the protective effects of translation or stall prone codons, it should be possible to do so.
The protective effects of ribosomal loading which could be modeled by making the decapping rate $\mu $ a function of $i$, e.g. $(1-i/\imax)$. Otherwise one could have a higher decapping rate for $m_0$ and a lower decapping rate for the other polysomal classes.
The protective effects of translation on decapping could increase per ribosome, but eventually at high \MRL could trigger ribosome associated decay pathways through ribosomal collisions, so $\mu$ would be a non monotonic function of $i$. This would require analysis on an individual transcript basis.
Our model does not consider codon specific effects such as pausing sites, difficult to fold regions of a protein or codon usage, or protein quality control \citep{RN39}.
Pausing sites could be addressed by splitting each polysome class into two regions and could approximate a ribosome flow model of only two regions, a 5' and 3', split by the pausing site. Current models of translation focus mainly on the behavior of the average transcripts. However this ignores the changing population of mRNAs necessary for protein production. 
Developing a more quantitative understanding of how different factors affect a gene's mRNA stability and, in turn, protein expression, relevant to a wide range of applied molecular biology (e.g. the design of efficient heterologous genes expression and mRNA vaccines) \citep{RN40}.



%
%
%
%
%
%
%
%
%
%
%
%
%
%
%
%\begin{enumerate}
%\item Summary paragraph of results?
%
%
%\item mRNA decay not only regulates transcript abundance, but also reduces mean ribosomal load.
%	\begin{enumerate}
%		\item This reduction can be interpreted as a inhibiting the system from reaching the unobstructed steady state. This has previously been suggested but not demonstrated by (Reuveni 2011) and explored by (Valleriani 2011). 
%		\item This suggests that if left given enough time in the capped state each transcript species would reach a characteristic \MRL during a set period of time.
%		\item  
%
%	\end{enumerate}
%	\begin{enumerate}
%	\item	Points for results
%
%	\end{enumerate}
%
%\item In addition to the effects on the distribution of transcripts across polysomal classes, decapping rate along with the scaled translation initiation rate determine the distribution of mRNA across the two states. 
%	\begin{enumerate}	
%		\item Mature mRNA populations and their degradation has been modeled before, but in absence of translation and it's quality control effects (Cao and Parker 2001, Cao and Parker 2003, Wu 2013, Wu 2016, Zupanic 2016, reviewed in Ashworth 2019). 
%		\item Our model shows that even decapped populations of mRNAs can produce protein and can be the most abundant species of mRNA in the population. 
%		\item Through 5'P mRNA sequencing Pelechano 2015 can track degradation intermediates. They find that degradation products in the CDS follow a 3 nucleotide periodic pattern and compose 12.4 of all reads recovered. This indicates that a significant portion of the bulk mRNA population is undergoing cotranslational decay.
%		\item  Figure 9 shows that a significant amount of mRNA can be found in the decapped state.
%Most transcripts will have a scaled translation initiation rate below 0.01 and have most of the mRNA in the capped state. However, our model result is on a per gene basis. To estimate a the global proportion of decapped to capped mRNA will require a larger sample of marking, transcription and scaled translation initiation rates.
%	\end{enumerate}
%	\begin{enumerate}
%	\item	Points for results
%	\item mention pelechano in results briefly
%	\end{enumerate}
%
%
%\item	Co-translational decay may allow for decapped transcripts to provide substantial protein production.
%	\begin{enumerate}
%	\item A surprising result from our model is that genes with short half-lives can have almost half of their protein produced in the decapped state. 
%	\item This suggests that short lived transcripts can produce more protein than expected despite having a half-life comparable to the time it takes an average length protein to get translated.
%	
%	\end{enumerate}
%	\begin{enumerate}
%	\item	Points for results
%
%	\end{enumerate}
%
%
%\item	Model Limitations, extension and future work, modeling decay
%	\begin{enumerate}
%	\item The 5' mRNA decay pathway presented in our model most closely resembles that of co-translational decay, and in Arabidopisis 68\% of mRNAs are degraded this way (Sorenson 2018).
%		\item However, other mechanisms exist with different outcomes for translation. 3' decay would result in no ribosome terminating and abruptly remove a subset of transcripts and overall protein production from the pool. Endonucleolytic decay due to NGD or NMD would potentially allow ribosomes downstream to terminate but not those upstream (Urquidi-Camacho 2020, Merchante 2017).
%		\item Extension of the model to include these mechanisms is yet to be explored. However, it is safe to speculate that they would both result in a further reduction of \MRL.
%		\item Our model also used two conservative set of decapping rate estimates (Presnyak 2015 and Sorenson 2018). 
%	
%	\end{enumerate}
%	\begin{enumerate}
%	\item	Points for results
%
%	\end{enumerate}
%
%\item	Model extension and future work, Codon opt
%	\begin{enumerate}
%	\item Currently there is a debate whether mRNA stability is regulated primarily through the protective effects of ribosomal association or through the suboptimal codons causing ribosomal stalling and the subsequect ribosome associated decay pathways  (Chan 2018 elife).
%	\item Our model currently cannot distinguish between the protective effects of translation, codon effects or other decay pathways.
%	\item In the current implementation of the model we did not directly explore the protective effects of ribosomal loading. This could be first implemented by including a similar weighting term analogous to the wieght for the intiation rate of (1-i/\imax).
%	\item Our model does not consider codon specific effects such as pausing sites. 
%	\item Difficult to fold or translate regions of a transcript  could be further modeled by splitting each transcript species into two or three regions defined biologically by pausing sites. This would resemble a nested ribosome flow model withing our model structure. 
%	\item These hyptotheses could be tested by fitting out model to the Chan 2018 dataset.
%	\item The protective effects of translation could increase per ribosome, but eventually at high loads  could trigger ribosome associated decay pathways through ribosomal collisions. This would require analysis on an individual transcript basis.
%	\end{enumerate}
%
%\item Model Extension: Population level modeling with degradation
%	\begin{enumerate}
%	\item Modeling individual transcripts is of value to understand the mechanism of translation.
%	\item Our model does not account for limitations in the tRNA or ribosome pools.
%	\item Modeling of the all the species of transcript (Nanikashvili 2019, Raveh 2016) and taking to account ribsome availability have been done previously (Shah 2013), but without considering mRNA degradation.
%	\item Our model is extensible to modeling the full mRNA population of a sample.
%	\end{enumerate}	
%	\begin{enumerate}
%	\item	Points for results
%
%	\end{enumerate}
%
%\end{enumerate}
%
%
%
%
%
%
%
%\begin{enumerate}
%\item In this work we present a model which tracks mRNA populations and their association with ribosomes across two states, a translationally competent capped state and incompetent decapped state. 
%	\begin{enumerate}
%	\item As the process of translation is entirely dependent on its mRNA substrate to produce protein, understanding the underlying fluctuation in mRNA is crucial. 
%	\item We demonstrate that the process of marking an mRNA for degradation through decapping alters the total number of mRNA molecules in a system, their distribution between the two states as well as lowering the mean number of ribosomes on a transcript. 
%	\item A surprising outcome of this process is that, under certain biological conditions, a substantial amount of protein produced may arise from terminating ribosomes in the decapped state.
%	\end{enumerate}
%\end{enumerate}
%
%\begin{enumerate}
%\item The model recapitulates empirical measurements of translation.
%	\begin{enumerate}	
%	\item Despite the model itself not being fit to data, through use of empirically obtained parameter values it still displays ribosomal loading patterns that would be expected in polysome profiles. 
%	\item In polysome profiles all transcripts in a sample are separated based on the number of ribosomes translating on them.
%	\item The majority of transcripts are found with 1-10 ribosomes and strongly bias towards lower ribosomal loads (Weinberg 2016, Lokdarshi 2020).
%	\item Throughout the scaled translation initiation range we observe that, for a protein of average length and transcripts with a long half-life, the transcript is below a \MRL of 10 for the majority of the range. 
%	\item Moreover, Duc and Song 2018, only studied the most highly translated genes (850 out of 5,300 expressed genes). This means that a majority of translated genes fall below the scaled translation initiation range used in this study.  
%%morisaki uses 10xFLAG-TAG which they call spaghetti monster. Flag tag is 8aa long. so 80aa tag. Plus 1544aa long KDM5B, or 374aa long beta actin or 125aa long H2B. The average ribosomal load estimated was 5, 3 and 2.2 respectively. Converting them to \imax that would be (80+1544)/9= 180,(80+374)/9=  50 and (80+125)/9= 22. and the densities would be 0.0277, 0.06, and 0.1. Converting to model \imax of 40, 1.1118, 2.4 and 4 ribosomes each. 
%%Yan uses a 24xSunTag kif18b PP7 reporter. only the suntag and kif are protein coding. A suntag has a length of 24 aa and kif18b 852aa. so total length is 852 + 24*24 = 1482. \imax of 158. the relative frequency of mRNAs with ribosomes are an average of 12 with a range of 1 to 30.  that is a density of 0.076 +- 0.0063 to 0.19. in \imax 40 that is 3 +- 0.25 to 7.6. 
%%Wang also uses a 24x suntag, but with ODC (461aa) as the protein. 461+24*24= 1037. \imax = 115. with an average ribosomal load of 12 +- 1-37. density of 0.104 +- 0.009 to 0.322. in \imax 40 that is 4.16 +- 0.36 to 13
%%Finally wu usese a 1x flag 24xsuntag blue fluorescent protein (239) and an auxin induced degron (228). total length is 1051. \imax is 116. mean of 5. 0.043 and at \imax 40 1.7.shorter and longer constructs allowed wu to see that number of ribsomes linearly scale with construct length .  agreeing with our results and indicating that their estimates are very likely steady state estimates.
%
%	\item This is further corroborated with single molecule imaging analyses. Rescaling ribosome abundances from each study to an \imax of 40 results in loads of 1, 2.4 and 4 ribosomes from (Morisaki 2016), 3 ribosomes (Yan 2016), 4 ribosomes (Wang 2016) and 1.6 (Wu 2016). 
%	\item Wu 2016, also designed constructs containing different lengths of protein sequences and found that ribosome load scales linearly with protein length. Agreeing with our results in Figure 8 and implying the measurement was performed at true steady state. 
%	\item Despite this general agreement, the model still represents an upper bound of translation as many other forms of translational control are not included. 
%	\end{enumerate}
%\end{enumerate}
%
%\begin{enumerate}
%	\item Our model integrates the effect of mRNA decay into translation.
%	\begin{enumerate} 
%		\item mRNA decay not only regulates transcript abundance, but also reduces mean ribosomal load.
%		\item This reduction can be interpreted as a inhibiting the system from reaching the unobstructed steady state. This has previously been suggested but not demonstrated by (Reuveni 2011) and explored by (Valleriani 2011). 
%		\item This suggests that if left unobstructed each transcript species would reach a characteristic \MRL during a set period of time. 
%		\item The 5' mRNA decay pathway presented in our model most closely resembles that of co-translational decay, and in Arabidopisis 68\% of mRNAs are degraded this way (Sorenson 2018).
%		\item However, other mechanisms exist with different outcomes for translation. 3' decay would result in no ribosome terminating and abruptly remove a subset of transcripts and overall protein production from the pool. Endonucleolytic decay due to NGD or NMD would potentially allow ribosomes downstream to terminate but not those upstream (Urquidi-Camacho 2020, Merchante 2017).
%		\item Extension of the model to include these mechanisms is yet to be explored. However, it is safe to speculate that they would both result in a further reduction of \MRL.
%		\item Our model also used two conservative set of decapping rate estimates (Presnyak 2015 and Sorenson 2018). 
%	\item Currently there is a debate whether mRNA stability is regulated primarily through the protective effects of ribosomal association or through the suboptimal codons causing ribosomal stalling and the subsequect ribosome associated decay pathways  (Chan 2018 elife).
%	\item Our model currently cannot distinguish between the protective effects of translation or 
%	\item In the current implementation of the model we did not directly explore the protective effects of ribosomal loading. This could be first implemented by including a similar weighting term analogous to the wieght for the intiation rate of (1-i/\imax).
%	\item However, biology suggests a more complex behavior.
%	\item The protective effects of translation could increase per ribosome, but eventually at high loads  could trigger ribosome associated decay pathways through ribosomal collisions. This leaves much to be explored further	
%	\end{enumerate}
%\end{enumerate}
%
%
%\begin{enumerate}
%	\item In addition to the effects on the distribution of transcripts across polysomal classes, decapping rate along with the scaled translation initiation rate determine the distribution of mRNA across the two states. 
%	\begin{enumerate}	
%		\item Mature mRNA populations and their degradation has been modeled before, but in absence of translation and it's quality control effects (Cao and Parker 2001, Cao and Parker 2003, Wu 2013, Wu 2016, Zupanic 2016, reviewed in Ashworth 2019). 
%		\item Our model shows that even decapped populations of mRNAs can produce protein and can be the most abundant species of mRNA in the population. 
%		\item Through 5'P mRNA sequencing Pelechano 2015 can track degradation intermediates. They find that degradation products in the CDS follow a 3 nucleotide periodic pattern and compose 12.4 of all reads recovered. This indicates that a significant portion of the bulk mRNA population is undergoing cotranslational decay.
%		\item  Figure 9 shows that a significant amount of mRNA can be found in the decapped state.
%Most transcripts will have a scaled translation initiation rate below 0.01 and have most of the mRNA in the capped state. However, our model result is on a per gene basis. To estimate a the global proportion of decapped to capped mRNA will require a larger sample of marking, transcription and scaled translation initiation rates.
%	\end{enumerate}
%\end{enumerate}
%
%\begin{enumerate}
%	\item Co-translational decay may allow for decapped transcripts to provide substantial protein production. 	
%	\begin{enumerate}
%		\item A surprising result from our model is that genes with short half-lives can have almost half of their protein produced in the decapped state. 
%		\item This suggests that short lived transcripts can produce more protein than expected despite having a half-life comparable to the time it takes an average length protein to get translated.
%	\end{enumerate}
%		\item We also highlight that all examples in the model have been run under the same transcription rate $\lambda$. The underlying amount of transcript will scale protein output globally, which translation and mRNA degradation will subsequently tune.
%	\begin{enumerate}
%		\item We have not correlated protein production to transcription as in (de Sousa Abreu 2013, Schwanhauser 2013, Edfors 2016, Brion 2020) due to not modeling protein degradation. 
%		\item Properly setting the transcription rate is important for determining final protein production, but not for understanding the translational dynamics.
%	\end{enumerate}
%\item In our model the remaining transcripts in class $m_0^*$ are either only the 3' end of co-translationally degraded transcripts or full transcripts from class $m_0$.
%	\begin{enumerate}
%		\item The role of the degradation rate  is to account for the remaining transcripts.
%		\item Mathematically and biologically it is essential for completing the lifecyle of an mRNA, but difficult to measure.
%	\end{enumerate}
%\end{enumerate}
%
%\begin{enumerate}
%	\item The model suggest that regulation mechanisms for translation are concerned with reducing \MRL.
%	\begin{enumerate}	
%		\item This is first seen by comparing DII, the simplest, most unobstructed model to the DDI model. 
%		\item As scaled translation initiation rates increase, density dependence reduces \MRL. We do not include bottlenecking due to translational folding, codon optimality or pausing in our model therefor density dependent effects might occur earlier in a transcript. 
%		\item With in the design of our model this would mean a transcript would behave as if its total length were <\imax. Given a shorter effective \imax for translation.
%		\item This particular aspect could be further modeled by splitting each transcript species into two or three regions defined biologically by pausing sites. This would resemble a nested ribosome flow model withing our model structure. 
%		\item As described above, mRNA degradation further reduces the \MRL as can mRNA secondary structure.
%		\item Other methods such as the canonical global repression of translation by eIF2$\alpha$ phoshorylation and upstream open reading frames are all inhibitory methods (Dever 2022, Lokdarshi 2020).
%		\item Finally, modeling of the all the species of transcript (Nanikashvili 2019, Raveh 2016) and taking to account ribsome availability have been done previously (Shah 2013), but without considering mRNA degradation.
%		\item All these biological phenomena provide ample room for extension of our base model.
%	\end{enumerate}
%\end{enumerate}
% 


%most interesting idea I have at the moment is the idea of effective \imax. While our model does not explicitly account for any of the mechanisms or positions of the ribosomes on a transcript, it does predict behavior for hypothetical transcripts. We currently are running on the assumption that one of the determining parameters is the average elongation rate across a transcript. This very well might not be true as 

%Result 1: Longer genes must be translated less efficiently than shorter genes with especially if there is a maximal intitaion/elongation/translation rate. however longer genes might also be more sensitive.  Sensitivity of Translational regulation based on gene size and intiation rate. From results 


%Result 2: Explore the rate limiting effects of tau (geometric,arithmetic sequence dependence) and mu on mean ribosome loading.


%Result 3: decapped class has only a limited effect of the total distribution of mRNAs and overall protein production.


%Result 4: Is translation density dependent or independent? Under which conditions would density start having an effect. 

%Where is translation tuned to? low mid high Translation? Which parameters can have a greater effect. There has to be a maximum speed for all processes (initiation elongation). Test for robustness of protein output vs ranges of these parameters. Does having an early slow stretch of codons effectively shorten a gene's \imax?


%Header for result
%First sentence of paragraph
%last sentence of paragraph

%4-5 key  findings due on Monday
%outline due on Wednesday

%Definitions of load vs density vs efficiency vs production rate











%Tau prime is the rate at which ribosomes leave the transcript.
%For a transcript with one ribosome tau prime is the sum of all the individual steps it took the ribosome to get to the stop codon. Since we are not modeling the actual elongation process step by step (and thus mechanistically ignoring pausing or ribosome collisions) we can easily calculate this rate by multiplying the number of codons in gene i by the average translation elongation speed for gene i. 
%For example in a 900 nucleotide long gene will produce a 300 amino acid long protein. If the average elongation rate of the protein is 3 aa/sec, the translation of one protein will take 300aa/3aa/s = 100 s.
%100 seconds is the residence time of a ribosomes on transcript j of gene i. If there are 2 ribosomes the residency time will halve and be 50s. 3 ribosomes = 33.33s, 4 25s and so on. 
%To get the rate of ribosomes leaving the transcript then we will take the reciprocal of this. The leaving rate for 1 ribosome is 1/100 or 0.01. For 2 it is 0.02, for 3, 0.03, and4 0.04 and so on. If every codon was populated by a ribosome (which is physically impossible since a ribosome occupies roughly 9 codons), then the leaving rate would equal the average elongation rate. However, the maximal leaving rate is now \imax*termination rate of 1 ribosome. i.e. (\imax/#codons)*tau. Or  1/9 tau
%The equation for the residence time is:
%τ'=codons/i*τfor i=1,2,…,\imax
%The equation for the termination rate is:
%\tau^\ast=1/\tau^'=i\ast\tau/\left(#codons\right);\ for\ i=0,1,2,\ldots,i_{max}
%Where plain tau is the average elongation rate for transcript j.






%add new decapped class results here. decapped class in its current formulation is very unpopulated. 


	
%work with an protein with a short medium and long lifespan, low medium and high intitiation rates, low medium high elongation rates and short, medium long length.

%What are some of the important interesting questions.
%What does the parameter space look like?
%	What are the currently known ranges for the parameters of interest and what ribosome densities do they produce.
%		Mu will follow the distribution of half lives found in an organims of  
%	Create parameter range which is bounded within relistic parameter values. What are the resulting ribosome loads?
%	For more extreme cases what would the parameters have to be? 
%	How does \imax affect the loading?
%	How does kappa interference affect loading?
%
%Half lives. What does each individual process contribute to the half life of transcripts? 
%	capped 
%	decapped 
%		M0* is delta driven
%		All other M* have a set behaivior dependent on the distribution of transcripts in capped.
%What processes does our current model mimic? What does it miss out? 


 


\section{Appendix}

%%%%%%%%%%%%%
In this section we present the matrix vector formulation of the model used to obtain both the analytical and numerical solutions in the main text. Additionally, we describe the analytical solution to the capped system. 
%Note that from here forward the capped and decapped subsystems are presented separately, the benefits of taking this approach will become evident as we proceed.  


\subsection{Matrix-vector Formulation of ODE System}
It is frequently useful to work with the matrix-vector formulation for a system of ODE.
In this model, the dynamics of the decapped and capped mRNAs can be represented as,
\begin{equation}
\vec{M}'=\boldsymbol{F}\vec{M}+\vec{B},
\end{equation} 
where $\vec{M}\in\mathbb{R}^{2(\imax+1)}$ is a vector of all state variables, ordered here as $m_0$, $m_1$, ..., $m_{\imax}$, $m^*_0$, $m^*_1$, ..., $m^*_{\imax}$, $\vec{M}'$ is the vector containing the first derivatives of $\vec{M}$ with respect to time, $\bs{F}\in\mathbb{R}^{2(\imax+1)\times 2(\imax+1)}$ is the matrix representing the full model (Equation\ref{eq:full_matrix}), and $\vec{B}\in\mathbb{R}^{2(\imax+1)}$ is the vector of $\lambda$ as the first component and 0s else.
Using the functional forms presented above, matrix formulations are provided next.

As opposed to explicitly listing elements of the full model matrix-vector representation we found that it is more convenient to utilize the block structure that emerges in this system and explicitly provide the block components.
The matrix $\bs{F}$ is block lower-diagonal and is given in Equation\ref{eq:full_matrix}.
\begin{equation}
\bs{F}=\left(\begin{array}{cc}
\bs{U} & \bs{0} \\
\bs{\mu} & \bs{R} \\
\end{array} \right).
\end{equation}
The upper-left block, $\bs{U}$, corresponds to the capped state variables, where $\bs{U}$'s general form is provided in Equation\ref{eq:capped_matrix}.
The upper-right block is a matrix of all zeros, $\bs{0}\in\mathbb{R}^{\imax+1\times \imax+1}$.
Using $\bs{I}$ to represent the $\imax+1\times \imax+1$ identity matrix, the lower-left block is $\bs{\mu}=\mu_0\bs{I}$, a diagonal matrix with the constant $\mu_0$ on the diagonal and 0s else.
The lower-right block, $\bs{R}$, corresponds to the decapped state variables and its form is provided in Equation\ref{eq:decapped_matrix}.

The matrix $\bs{U}$ is $(\imax+1\times \imax+1)$ dimensional and is tri-diagonal with non-zero entries on the diagonal, super-, and sub-diagonals,
\pagebreak
\begin{strip}
\begin{equation}
\bs{U}=\left(\begin{array}{cccccc}
-(\kappa_0+\mu_0) & \tau_0\frac{1}{\imax} &  &  &  & \\
\kappa_0 & \left(1-\frac{1}{\imax} \kappa_0+\mu_0+\tau_0\frac{1}{\imax}\right) & \tau_0\frac{2}{\imax} &  &  & \\
   &\ddots        & \ddots        & \ddots & &  \\
   & &    1-\frac{(i-1)}{\imax}\kappa_0 & -\left(1-\frac{i}{\imax}\kappa_0+\mu_0+\tau_0\frac{i}{\imax}\right) & \tau_0\frac{i+1}{\imax} & \\
                  &         &        & \ddots  & \ddots & \ddots \\
     
                          &        &  &  & \frac{1}{\imax}\kappa_0 & -\left(\mu_0+\tau_0\frac{\imax}{\imax}\right) \\
\end{array}\right).
\end{equation}

In the representation given in Equation\ref{eq:capped_matrix}, all blank entries are 0.
The $(\imax-1)^{\text{th}}$ row has been suppressed in Equation\ref{eq:capped_matrix}, but it can be generated using the formula included for the $i^{th}$ row.

The matrix $\bs{R}$ is the lower-right block in the block lower-diagonal matrix $\bs{F}$ (Equation\ref{eq:full_matrix}),
\begin{equation}
\bs{R}=\left(
\begin{array}{ccccccc}
-\delta & \tau_0\frac{`}{\imax} & & & & & \\
 & -\tau_0\frac{1}{\imax} & \tau_0\frac{2}{\imax} & & & &\\
 & & \ddots & \ddots & & & \\
 & & & -\tau_0\frac{i-1}{\imax} & \tau_0\frac{(i+1)}{\imax} & & \\
 & & & & \ddots & \ddots & \\
 & & & & & -\tau_0\frac{(\imax-2)}{\imax} & \tau_0\frac{\imax}{\imax} \\
 & & & & & & -\tau_0\frac{\imax}{\imax} \\
\end{array}
\right),
\end{equation}
\end{strip}
$\bs{R}$ is upper-diagonal with only non-zero entries on the diagonal and the super-diagonal.

\subsubsection{Capped Subsystem Matrix-vector Representation}
As a group the capped subsystem decouples from the decapped subsystem, as such the capped subsystem can be solved independently of the decapped subsystem.  
The matrix-vector formula representing the capped subsystem is \begin{equation}
\vec{m}'=\bs{U}\vec{m}+\vec{b},\end{equation} where $\vec{m}\in\mathbb{R}^{\imax+1}$ is the vector of capped state variables ordered $m_0$, ..., $m_{\imax}$, $\vec{m}'$ is the vector containing the first derivatives of $\vec{m}$ with respect to time, $\bs{U}\in\mathbb{R}^{\imax+1\times \imax+1}$ is the matrix representing the capped subsystem (Figure\ref{eq:capped_matrix}), and $\vec{b}\in\mathbb{R}^{\imax+1}$ is the vector of $\lambda$ as the first component and 0s else.
With all equations defined for the full ODE system, include matrix-vector representations, the next section outlines methods for finding steady-state solutions to the system.


\subsubsection{Capped state steady state solution}

The capped system can be split into two components: Total transcripts in the capped state and how the transcripts are distributed across polysome classes. 
From manual exploration of model solutions of the capped state at low \imax values. 
We discovered that the capped class transcript number is determined by $\lambda/ \mu$
	
The solution to the system, as presented previously, can be expressed in the determinant-adjoint form:
	\begin{equation*}
		\vec{m}=-\frac{1}{\det[\bs{U}]}Adj[\bs{U}]\vec{b}.
	\end{equation*}
As $\vec{b}$ is [$\lambda$ 0 0 0 ... 0]. Only the first column of the adjoint matrix contributes to the result. 
	\begin{equation*}
		Adj[\bs{U}]\vec{b} = \lambda\vec{a}
	\end{equation*}	
and
	\begin{equation*}
		\sum_{j=0}^{\imax}\vec{a}_j = a_{tot} 
	\end{equation*}
With this we can factor our solution into two parts: 1) the total transcript abundance and 2) The distribution of transcript across the polysome classes.
	\begin{equation*}
		\vec{m}=-\frac{\lambda a_{tot}}{\det[\bs{U}]} \frac{\vec{a}}{a_{tot}} 
	\end{equation*}
Where:
	\begin{equation*}
		\frac{\vec{a}}{a_{tot}} = \vec{p}_m
	\end{equation*}
The vector $\vec{p}_m$ sums to one and contains the probabilities of finding and mRNA in each class in the capped state. Now we are left with
	\begin{equation*}
		\vec{m}=-\frac{a_{tot}}{\det[\bs{U}]} \: \lambda\vec{p}_m
	\end{equation*}
If we sum across all classes to get the total mRNA population we find,
	\begin{equation*}
		\sum_{i=0}^{\imax}m_{i} =-\sum_{i=0}^{\imax} \frac{a_{tot}}{\det[\bs{U}]} \: \lambda\vec{p}_m =-\frac{a_{tot}}{\det[\bs{U}]} \: \lambda = \frac{\lambda}{\mu}
	\end{equation*}
	\begin{equation*}
		-\frac{a_{tot}}{\det[\bs{U}]} = \frac{1}{\mu}
	\end{equation*}
We finally arrive at,
	\begin{equation} 
		\vec{m}=\frac{\lambda}{\mu}\vec{p}_m
	\end{equation}


The terms on the left hand side of the equation represent the total transcript population. The right hand side is the vector of probabilities, one entry for each class and is a function of $\kappa$, $\tau$, and $\mu$.	


%\bibliographystyle{NAR-natbib}
\bibliographystyle{plain}
\bibliography{Modeling_RNA_Populations_2}

%\printbibliography

%\subsection{Additional text}
% Presnyak utilized the temperature sensitive \textit{rbp-1} RNA polymerase mutant in yeast. This mutant can not undergo transcription at non-optimal temperatures, thus allowing for the measurement of mRNA decay over time. Sorenson (2018) used the transcriptional inhibitor, cordycepin, to treat  \textit{Arabidopsis thaliana} seedlings and measured their decay using RNA-Seq.
%
%.This is due to the fact that many combinations of $\kappa$ and $\tau$ can yield the same scaled translation initiation rate (e.g. $\kappa$ =0.02 and $\tau*9*\imax$ = 2, and $\kappa$=0.04 and $\tau*9*\imax$=4, both yield $\kappa/\tau$=0.01).
%
%
%	
% Recently, the rate of degradation for the 5' - 3' exonuclease XRN1 was determined to be ~26 nt/s (Atthapattu 2021). XRN1 is the primary exonuclease involved in co-translational degradation and  5' degradation pathways (Sorenson 2018,Yu 2016, Collart 2019, Pelechano 2015). For an average 3' UTR of 121 nts (Kebaara 2009) this would take ~4.6s, and an average transcript of ~ 1400nt would take 54s to degrade.  This means the average mRNA clearance rate $\delta$ ranges from 0.019/s to 0.22/s, or approximately 1 order of magnitude. The total population of mRNA in $\msumstar$ is determined by $1/\delta$, $1/\tau$ and $\mu$ (as part of $\vec{S}$) as shown Equation  \ref{eq:  marked_total_pop}. $\tau$ ranges from 0.03 to $10^{-4}$. This makes $1/\delta \leq 1/\tau$. It is reasonable to explore the model with large $\delta$ since decapped transcripts are translationally incompetent. 

% without interference if kappa is small relative to tau then interference doesn't matter. if kappa is large what is the distribution? How does increasing mu affect it.

%
%\newpage
%
%\part{Interpreting Data and Mapping the Model to Experimental Data}
%
%
%%\section{Determination of Solutions at Steady-State in Both the Continuous
%%and Discrete Models}
%
%%({*}{*} This should include some of the writing above, including the
%%techniques discussed that utilized the tridiagonal form of the matrix
%%and also using the determinant-adjoint technique for inverting the
%%matrix. {*}{*})
%
%
%\section{Interpreting Experimental Data}
%
%This section will discuss the technique used to interpret the experimental
%data for the distribution of polysomes.  The data being considered here is the work of \emph{J. Guan,} and \emph{A. VonArnim , University of Tennesse - Knoxville}.
%
%
%\subsection{Absorbance Data}
%
%Absorbance data has been generated by a variety of researchers to
%quantify the abundance of different classes of polysomes. This data is often broken down into polysomal
%fractions. The number of fractions can vary depending on the researcher;
%10 to 12 fractions is fairly common in the literature. It is also
%customary to lump the first fractions together into a single non-polysomal
%class. This class includes bare mRNA, those that have ribosomes still
%in the process of binding completely for translation, and those that
%have a single ribosome bound. This non-polysomal class can have up
%to four fractions included in it. A similiar phenonmenon takes place
%in the final fractions. Theoretically, the maximum number of ribosomes
%that can be bound is related to the length of the mRNA. This means
%there can be upwards of forty different mRNA classes (this number
%is related to \imax in the discrete model and $n$ in the continuous
%model). So with 12 fractions and 40 mRNA classes, there is necessarily
%an overlap of classes in each fraction.
%
%The absorbance data itself is reported as a the $log_{2}$ of integral
%of absorbance in each fraction.
%
%
%\subsection{Data Interpretation Techniques}
%
%In interpreting the data there are multiple elements that need to
%be considered first before the abundances estimated by the model can
%be matched up with the UV profiles. A function to relate mRNA class
%to the mean distance travelled in the profile is one piece of information
%that is needed. Also, an estimate of the variance in this travel distance
%is also needed. If a function is to be written to estimate these values,
%one must also know the total length of the UV-Profile, and the background,
%or baseline, amount of fluoresence.
%
%
%\subsubsection{Assumption of Normality and Determiniation of Distribution Parameters}

%If we assume that the distribution of mRNA of a given class in the
%UV profile is normally distributed with mean $\mu$and variance $\sigma$,
%then there are graphical techniques that can be used to determine
%these constants directly from UV-Profile diagrams.
%
%Mean distance traveled in the UV Profile is estimated by measuring
%the distance of the peaks in the profile from the ``origin'' in
%the profile (where the origin represents 0\% absorbance and 0 distance
%travelled). This was done by importing the PDF image of the UV-Profile
%into \emph{Mathematica} and then using the standard \emph{'get-coordinates'}
%graphing tool. This work took place under the asssumption that the
%first visible peak in the UV-Profile was from the mRNA class with
%2 ribosomes bound. The coordinates of each distinguishable peak thereafter
%was also found and this data was used to fit a saturating function
%whose maximum value was given by the total length of the UV-Profile.
%The function fitting was performed in \emph{Mathematica }using \emph{'FindFit'}.
%
%The variance was determined by gathering points from the different
%peaks and utilizing the property of the Normal distribution that the
%2nd derivative is related to the curvature of the function. A quadratic
%function was fit to each peak and then the second deriative was found
%and equated to the second derivative of the probability distribution
%function of the Normal distribution. In this way, the distribution
%parameter $\sigma$ was able to be found. Then with estimates of $\sigma$from
%each distinguishable peak in the profile, a function was fit in order
%to estimate $\sigma$ as function of the polysomal class.
%
%Using the estimated values for mean and variance, denoted $\mu_{est}(i)$
%and $\sigma_{est}(i)$ respectively, a mixed PDF was created to mimic
%the UV-Profile. The mixed PDF is comprised of the normal distributions
%$N_{i}(\mu_{est}(i),\sigma_{est}(i))$ and is denoted as $M_{pdf}(N_{i}).$
%
%
%\subsubsection{Connecting the Model to Absorbance Data}
%
%With a function created to mimic the UV-Profile image, the next step
%is to use the absorbance Data to estimate parameters in the model.
%The assumption is made that the height of each peak (and necessarily
%the $log_{2}$of the integral in a fraction) in the UV-Profile is
%proportional to the abundance of each polysomal class. The distribution
%of these abundances is what the model predicts, and so within the
%model we see that the fluoresence in a fraction can be viewed as a
%function of the model parameters and the distribution parameters.
%Letting $\Lambda$represent the vector of model parameters, and $Y_{j}$
%represent the measured absorbance data in fraction $j,$we have
%the following relationship:
%\[
%Y_{j}\propto log_{2}(\sum_{i}\int_{j}^{j+1}\msum(i,\Lambda)\, N_{i}(\mu_{est}(i),\sigma_{est}(i))di)
%\]
%
%
%This states that the amount of absorbance in fraction $j$ is proportional
%to the integral of each Normal function on that fraction times the
%abundance calculated from the model, since each normal function is
%not isolated completely to a fraction, we need to sum the contributions
%from all the functions, which is why the summation over all values
%of $i$ is needed in the relationship. This proportionality is lays
%the framework for the minimization and estimation of the model parameters,
%$\Lambda.$
%
%
%\subsection{Minimization for Parameter Estimation}
%
%This section will present the motivation for the minimization technique
%utilized in the research thusfar. Based on the assumption of Normally
%distributed error (or noise) in the measurement, minimization of the
%sums of difference in model predictions and measurement squared is
%used. This method maximizes Log-Likelihood for the parameter estimates.
 %
%
%
%\subsubsection{Minimization Scheme}
%
%Definitions of the functions and variables used will be given below.
%Beginning with the definitions and relationship relating the data
%to the model estimations in a fluorsecence fraction, where $j\in J$:
%\begin{description}
%\item [{$f_{j}$:}] True value of absorbance in $j^{th}$ fraction, without
%noise in measurement. This is related to model estimates. When considering
%model estimates of this, it will be written as $(f_{j}|\Lambda)$
%\item [{$F_{j}$:}] Value of absorbance in $j^{th}$ fraction with noise
%added.
%\item [{$\epsilon$:}] The amount measurement error or noise, the difference
%between $f_{j}$ and $F_{j}$
%\end{description}
%Related by the following, under the assumption that the noise in each
%fraction has the same distribution:
%
%\begin{align*}
%F_{j} &= f_{j}\cdot\hat{\epsilon}\\
%log_{2}(F_{j}) &= log_{2}(f_{j})+log_{2}(\hat{\epsilon})\\
%Where & \, & \hat{\epsilon}\sim N(0,\sigma_{\epsilon})
%\end{align*}
%The data being considered is given in $log_{2}$of the absorbance
%in a fraction, so for convenience the following will be used:
%\begin{description}
%\item $G_{j}=Log_{2}(F_{j})$
%\item $g_{j}=Log_{2}(f_{j})$
%\item $\epsilon=Log_{2}(\hat{e})\sim LogNormal(0,\sigma_{\epsilon})$
%\end{description}
%Then, the likelihood of $G_{j}$, is given by the following:
%\begin{align*}
%Lik(G_{j}) &= \dfrac{1}{\sqrt{2\pi\sigma_{\epsilon}}}Exp(-\dfrac{(G_{j}-(g_{j}|\Lambda))^{2}}{2\sigma_{\epsilon}^{2}})
%\end{align*}
%
%
%and the $negative\; loglikelihood$ is given by:
%\begin{equation*}
%-LLik(G_{j})=-\ln(\dfrac{1}{\sqrt{2\pi}})+\ln(\sigma_{\epsilon})+\dfrac{(G_{j}-(g_{j}|\Lambda))^{2}}{2\sigma_{\epsilon}^{2}}
%\end{equation*}
%
%
%The objective function to minimize is then:
%\begin{equation*}
%\min_{j}\; n\ln(\sigma_{\epsilon})+\sum_{j\in J}\dfrac{(G_{j}-(g_{i}|\Lambda))^{2}}{2\sigma_{\epsilon}^{2}}
%\end{equation*}
%
%\section{Results}
%
%This section will include our results from simulation as they become available.
%\part*{Appendix}
%\section*{A1: Example solutions to capped system for ODE formulation at equilibrium}
%\begin{description}
%\item This set of solutions is generated from solving the system of capped ODE at equilibrium directly using \emph{Solve[]} in \emph{Mathematica}
%\item For $\imax=1$
%\item \hspace{5pt} $\left\{M(0)\to \frac{\lambda  (\mu +\tau )}{\mu  (\kappa +\mu +\tau )},M(1)\to \frac{\kappa  \lambda }{\mu  (\kappa +\mu +\tau )}\right\}$
%\item For $\imax=2$ 
%\item \hspace{5pt} $\left\{M(0)\to \frac{\lambda  (\kappa  \mu +(\mu +\tau ) (\mu +2 \tau ))}{\mu  \left(\kappa ^2+2 \kappa  (\mu +\tau )+(\mu +\tau ) (\mu +2 \tau )\right)},M(1)\to \frac{\kappa  \lambda  (\mu +2 \tau )}{\mu  \left(\kappa ^2+2 \kappa  (\mu +\tau )+(\mu +\tau ) (\mu +2 \tau )\right)},M(2)\to \frac{\kappa ^2 \lambda }{\mu  \left(\kappa ^2+2 \kappa  (\mu +\tau )+(\mu +\tau ) (\mu +2 \tau )\right)}\right\}$ 
%\item For $\imax=3$
%\item \hspace{5pt} $\Big\{ M(0)\to \frac{\lambda  \left(\kappa ^2 \mu +2 \kappa  \mu  (\mu +2 \tau )+(\mu +\tau ) (\mu +2 \tau ) (\mu +3 \tau )\right)}{\mu  \left(\tau ^2 (6 \kappa +11 \mu )+3 \tau  (\kappa +2 \mu ) (\kappa +\mu )+(\kappa +\mu )^3+6 \tau ^3\right)},M(1)\to \frac{\kappa  \lambda  (\kappa  \mu +(\mu +2 \tau ) (\mu +3 \tau ))}{\mu  \left(\tau ^2 (6 \kappa +11 \mu )+3 \tau  (\kappa +2 \mu ) (\kappa +\mu )+(\kappa +\mu )^3+6 \tau ^3\right)},$
%\item \hspace{9pt} $ M(2)\to \frac{\kappa ^2 \lambda  (\mu +3 \tau )}{\mu  \left(\tau ^2 (6 \kappa +11 \mu )+3 \tau  (\kappa +2 \mu ) (\kappa +\mu )+(\kappa +\mu )^3+6 \tau ^3\right)},M(3)\to \frac{\kappa ^3 \lambda }{\mu  \left(\tau ^2 (6 \kappa +11 \mu )+3 \tau  (\kappa +2 \mu ) (\kappa +\mu )+(\kappa +\mu )^3+6 \tau ^3\right)} \Big\} $
%\item For $\imax=4$
%\item \hspace{5pt} $\Big\{M(0)\to \frac{\lambda  \left(\mu  \tau ^2 (18 \kappa +35 \mu )+5 \mu  \tau  (\kappa +\mu ) (\kappa +2 \mu )+\mu  (\kappa +\mu )^3+50 \mu  \tau ^3+24 \tau ^4\right)}{\mu  \left(2 \tau ^3 (12 \kappa +25 \mu )+\tau ^2 (2 \kappa +5 \mu ) (6 \kappa +7 \mu )+2 \tau  (2 \kappa +5 \mu ) (\kappa +\mu )^2+(\kappa +\mu )^4+24 \tau ^4\right)},$
%\item \hspace{9pt} $M(1)\to \frac{\kappa  \lambda  \left(\kappa ^2 \mu +2 \kappa  \mu  (\mu +3 \tau )+(\mu +2 \tau ) (\mu +3 \tau ) (\mu +4 \tau )\right)}{\mu  \left(2 \tau ^3 (12 \kappa +25 \mu )+\tau ^2 (2 \kappa +5 \mu ) (6 \kappa +7 \mu )+2 \tau  (2 \kappa +5 \mu ) (\kappa +\mu )^2+(\kappa +\mu )^4+24 \tau ^4\right)},$
%\item \hspace{9pt} $M(2)\to \frac{\kappa ^2 \lambda  (\kappa  \mu +(\mu +3 \tau ) (\mu +4 \tau ))}{\mu  \left(2 \tau ^3 (12 \kappa +25 \mu )+\tau ^2 (2 \kappa +5 \mu ) (6 \kappa +7 \mu )+2 \tau  (2 \kappa +5 \mu ) (\kappa +\mu )^2+(\kappa +\mu )^4+24 \tau ^4\right)},$\item \hspace{9pt} $M(3)\to \frac{\kappa ^3 \lambda  (\mu +4 \tau )}{\mu  \left(2 \tau ^3 (12 \kappa +25 \mu )+\tau ^2 (2 \kappa +5 \mu ) (6 \kappa +7 \mu )+2 \tau  (2 \kappa +5 \mu ) (\kappa +\mu )^2+(\kappa +\mu )^4+24 \tau ^4\right)},$
%\item \hspace{9pt} $M(4)\to \frac{\kappa ^4 \lambda }{\mu  \left(2 \tau ^3 (12 \kappa +25 \mu )+\tau ^2 (2 \kappa +5 \mu ) (6 \kappa +7 \mu )+2 \tau  (2 \kappa +5 \mu ) (\kappa +\mu )^2+(\kappa +\mu )^4+24 \tau ^4\right)}\Big\}$
%
%\item Recall that the decapped solutions are solved completely in terms of the capped solutions using equation (7).
%
%\end{description}





\end{document}
